\documentclass[12pt]{article}
\usepackage{zocal}


\begin{document}

\begin{titlepage}
    \begin{center}
        \textbf{\textnormal{Escola Politécnica-USP}}\\
        \textbf{\textnormal{LARC}}

        \vspace{3cm}
        
         \includegraphics[width=0.5\textwidth]{images/Logo-Escola-Politécnica-Minerva.pdf}\\[1cm]
        
        {\Huge Seminário NTT}\\[1.5cm]
        \vspace{2cm}
        
        \textbf{Resumo}\\[0.5cm]

    \end{center}
\end{titlepage}
\thispagestyle{empty}

% --- SUMÁRIO ---
\tableofcontents
\newpage




\section{ A Dualidade entre Tempo e Frequência}
A Transformada de Fourier é uma operação matemática que mapeia uma função do domínio do tempo (ou espaço) para o seu domínio dual: a frequência. Essa transição é extremamente útil, pois propriedades que são complexas de analisar no tempo tornam-se claras no espectro de frequências.


Esta ferramenta é um pilar fundamental em diversas áreas do conhecimento:

\begin{itemize}
    \item \textbf{Matemática Pura:} Essencial na Teoria Analítica dos Números e no estudo de Equações Diferenciais Parciais (EDPs).
    \item \textbf{Física Moderna:} É a base matemática do \textbf{Princípio da Incerteza de Heisenberg} na Mecânica Quântica, onde a posição e o momento de uma partícula formam um par de variáveis conjugadas de Fourier.
    \item \textbf{Engenharia:} Processamento de sinais, compressão de dados (MP3, JPEG) e telecomunicações.
\end{itemize}


\section{ Transformada de Fourier Contínua (CTFT)}
Para uma função contínua $g(t)$, a transformada é definida pela integral:

\[\mathcal{F}(f) = \int_{-\infty}^{\infty} g(t) e^{-2\pi if t} \, dt\]

Ela pode ser entendida como um produto interno(uma projeção) do sinal com todos as frequencias da reta real $ \langle g, e^{-2\pi i ft} \rangle$, que, devido a ortoganalidade das frequências diferentes e que funcoes bem comportadas podem ser decompostas em series de autofuncoes $e^{-ift}$, consegue extrair extamente as frequências do sinal.
Apesar de sua elegância teórica, a CTFT apresenta desafios para a aplicação prática em sistemas digitais:

\begin{enumerate}
    \item \textbf{Natureza Analítica:} A resolução de integrais impróprias exige uma manipulação simbólica que é difícil de implementarz em computadores comuns.
    \item \textbf{Limite Infinito:} A definição pressupõe que conhecemos o sinal de $-\infty$ a $+\infty$, o que é impossível em cenários reais.
    \item \textbf{Amostragem Finita:} Na prática, os sinais são capturados de forma discreta (amostras) e por um tempo limitado, o que torna a integral contínua inaplicável.
\end{enumerate}


\section{ Transformada Discreta de Fourier (DFT)}
Para viabilizar o processamento em computadores, utilizamos a \textbf{DFT}. Ela opera sobre uma sequência finita de $N$ amostras, mapeando dados discretos no tempo para dados discretos na frequência:

\[X[k] = \sum_{n=0}^{N-1} x[n] e^{-i \frac{2\pi}{N} kn}\]

Para $k = 0, 1, \dots, N-1$.

Diferente da versão contínua, a DFT lida com somatórios e vetores numéricos, permitindo que a teoria de Fourier seja aplicada em qualquer dispositivo digital.
Eh possivel provar que a DFT eh um transformacao linear, logo, pode ser representada matricialmente.

Seja $\zeta_N = e^{-i\frac{2\pi}{N}}$. A representação matricial da DFT para $n = 0, 1, \dots, N-1$ é:

\[
\begin{bmatrix} 
X[0] \\ X[1] \\ \vdots \\ X[N-1] 
\end{bmatrix}
=
\begin{bmatrix}
1 & 1 & 1 & \dots & 1 \\
1 & \zeta_N^{1} & \zeta_N^{2} & \dots & \zeta_N^{N-1} \\
1 & \zeta_N^{2} & \zeta_N^{4} & \dots & \zeta_N^{2(N-1)} \\
\vdots & \vdots & \vdots & \ddots & \vdots \\
1 & \zeta_N^{N-1} & \zeta_N^{2(N-1)} & \dots & \zeta_N^{(N-1)(N-1)}
\end{bmatrix}
\begin{bmatrix} 
x[0] \\ x[1] \\ \vdots \\ x[N-1] 
\end{bmatrix}
\]

\subsection{Definição da IDFT}
A reconstrução do sinal original no domínio do tempo a partir de suas amostras de frequência é realizada pela IDFT:

\[
x[n] = \frac{1}{N} \sum_{k=0}^{N-1} X[k] \zeta_N^{-nk}, \quad n = 0, 1, \dots, N-1
\]

Onde $\zeta_N^{-nk} = e^{i \frac{2\pi}{N} nk}$. Matricialmente, a IDFT é dada por:

\[
\begin{bmatrix} 
x[0] \\ x[1] \\ x[2] \\ \vdots \\ x[N-1] 
\end{bmatrix}
=
\frac{1}{N}
\begin{bmatrix}
1 & 1 & 1 & \dots & 1 \\
1 & \zeta_N^{-1} & \zeta_N^{-2} & \dots & \zeta_N^{-(N-1)} \\
1 & \zeta_N^{-2} & \zeta_N^{-4} & \dots & \zeta_N^{-2(N-1)} \\
\vdots & \vdots & \vdots & \ddots & \vdots \\
1 & \zeta_N^{-(N-1)} & \zeta_N^{-2(N-1)} & \dots & \zeta_N^{-(N-1)(N-1)}
\end{bmatrix}
\begin{bmatrix} 
X[0] \\ X[1] \\ X[2] \\ \vdots \\ X[N-1] 
\end{bmatrix}
\]
Percebe-se que nesse caso discreto, a DFT e IDFT atuam como uma matriz mudanca de base, saindo da base do tempo e indo para base das raizes unitarias

A DFT pde ser vista de outra óptica, como avalição de polinomios.
Seja
\[
a(x)=\sum_{n=0}^{N-1} a_n x^n
\]
e tome $\zeta_N=e^{-\frac{2\pi i}{N}}$ seja uma $N$-ésima raiz primitiva da unidade. Define
\[
A_k := a(\zeta_N^k).
\]
Então 
\[
A_k
= \sum_{n=0}^{N-1} a_n(\zeta_N^k)^n
= \sum_{n=0}^{N-1} a_n\,\zeta_N^{kn},
\]
o que é exatamente a formulação da DFT, para $a_n \equiv x[n]$ e $A_n \equiv X[n]$, opostamente, a IDFT é interpretada como interpolção de polinômios:

Dados os valores
\(\{A_k\}_{k=0}^{N-1}\), ela reconstrói os coeficientes \(\{a_n\}_{n=0}^{N-1}\) do
único polinômio de grau \(N-1\) que satisfaz \(a(\zeta_N^k)=A_k\) para todo \(k\).
Explicitamente,
\[
a_n=\frac{1}{N}\sum_{k=0}^{N-1} A_k\,\zeta_N^{-kn}.
\]
Assim, \(\mathrm{DFT}\) é \emph{avaliar} em raízes da unidade e \(\mathrm{IDFT}\) é
\emph{interpolar} (recuperar os coeficientes) a partir dessas avaliações. (Essa perspectiva baseia-se na determinição unívoca do polinómio de grau $N-1$ por N pontos veja a seção~\ref{sec:Determinição unívoca do polinômio} para demonstração desse fato)


\section{ A Multiplicação de Polinômios e a Complexidade Computacional}

Um problema simplificado pela mudanca de domino eh a multiplicacao de polinomios. Tome os polinomios $f(x)$ e $g(x)$ de grau $n-1$: 

\[f(x) = \sum_{i=0}^{N-1} a_i x^i, \quad g(x) = \sum_{j=0}^{N-1} b_j x^j\]

Na abordagem clássica , o produto $h(x) = f(x) \cdot g(x)$ é obtido distribuindo-se cada termo de $f$ sobre todos os termos de $g$. Este processo resulta em um novo polinômio de grau $2N-2$:

\[h(x) = \sum_{k=0}^{2N-2} c_k x^k\]
onde $c_k = \sum_{i+j=k} a_i b_j$.

Nesta metodologia, o cálculo de cada coeficiente $c_k$ exige múltiplas operações de produto e soma, resultando em uma complexidade assintótica $O(n^2)$. Para polinômios com grandes volumes de coeficientes, este custo computacional torna o metodo inviavel.



A conexão entre a multiplicação de polinômios e a análise de Fourier vem do fato de que,
se $h(x)=f(x)g(x)$, então os coeficientes $c_k$ de $h$ são dados pela \textbf{convolução linear}
dos coeficientes de $f$ e $g$:
\[
c_k=\sum_{i+j=k} a_i\, b_{j}.
\]

Neste trabalho, devido ao foco na NTT e ao anel quociente $\mathbb{Z}_p[x]/(x^n-1)$,
trabalhamos com a \textbf{convolução circular} (de comprimento $n$), definida por
\[
c_k=\sum_{i=0}^{N-1} a_i\, b_{(k-i)\bmod n}, \qquad k=0,\dots,N-1.
\]

\textbf{Teorema da Convolução:} a transformada de uma convolução no domínio do tempo (ou espaço)
é o produto ponto a ponto (Hadamard) das transformadas no domínio da frequência:
\[
\mathcal{F}(f*g)=\mathcal{F}(f)\odot \mathcal{F}(g).
\]

Assim, o cálculo custoso da convolução  é convertido em um produto ponto a ponto,
pois a base de Fourier diagonaliza o operador de convolução (circulante).
Com a transformada direta ingênua o custo ainda é $O(n^2)$, não há ganho da multiplicação clássica de polinômios.

Uma exemplo para o convencimento do leitor foi disposto no apendice~\ref{sec:convolução}
\section{ A Fast Fourier Transform}

A FFT (Fast Fourier Transform) eh uma maneira de otimizar o calculo da DFT.

O algoritmo da FFT foi redescoberta por Cooley e Tukey em 1965, uma vez que Gauss ja tinha utilzado um algoritmo semelhante para calcular a orbitas de asteroides em 1805. 

O algoritmo se baseia em \textbf{dividir para conquistar}.

\begin{quote}
Relembrando
\end{quote}

As raízes unitárias possuem propriedades cíclicas e certas simetrias que permitem a economia nos cálculos, vejamos um exemplo.
\[\zeta_4^1 = e^{-i\frac{2\pi}{4}} = e^{-i90^\circ} = -i\]

\[
\begin{aligned}
\zeta^1 &= -i \qquad\qquad
\zeta^2 = -1 \\[6pt]
\zeta^3 &=  i \qquad\qquad
\zeta^4 = 1
\end{aligned}
\]
por isso, percebe-se que, a cada 2 "deslocamentos", o valor se torna o oposto, como ilustrado na figura:

%verificar se usar com duas raizes ou apenas a 4-esima raiz
%raizes
\begin{figure}[H]
\centering

% --- Subfigura 1: 4-ésimas raízes ---
\begin{subfigure}[b]{0.48\linewidth}
    \centering
    \begin{tikzpicture}[scale=1.8]
        \def\R{1}
        \draw[gray, thin, dashed] (0,0) circle (\R);
        \draw[->, >=Stealth, gray!50] (-1.2,0) -- (1.2,0) node[right] {\tiny Re};
        \draw[->, >=Stealth, gray!50] (0,-1.2) -- (0,1.2) node[above] {\tiny Im};

        \foreach \k in {0,1,2,3} {
            \coordinate (P\k) at ({\k*360/4}:\R);
            \fill (P\k) circle (1.5pt);
        }

        \node[above right] at (P0) {\tiny $1$};
        \node[above right] at (P1) {\tiny $\zeta = i$};
        \node[above left]  at (P2) {\tiny $\zeta^2 = -1$};
        \node[below left]  at (P3) {\tiny $\zeta^3 = -i$};

        \draw[blue!30, thin] (P0) -- (P2);
        \draw[blue!30, thin] (P1) -- (P3);
    \end{tikzpicture}
    \caption{4-ésimas raízes da unidade}
    \label{fig:raizes_4}
\end{subfigure}
\hfill
% --- Subfigura 2: 8-ésimas raízes ---
\begin{subfigure}[b]{0.48\linewidth}
    \centering
    \begin{tikzpicture}[scale=1.8]
        \def\R{1}
        \draw[gray, thin, dashed] (0,0) circle (\R);
        \draw[->, >=Stealth, gray!50] (-1.2,0) -- (1.2,0) node[right] {\tiny Re};
        \draw[->, >=Stealth, gray!50] (0,-1.2) -- (0,1.2) node[above] {\tiny Im};

        \foreach \k in {0,...,7} {
            \coordinate (P\k) at ({\k*360/8}:\R);
            \fill (P\k) circle (1.5pt);
        }

        \node[above right] at (P0) {\tiny $1$};
        \node[above right] at (P1) {\tiny $\zeta$};
        \node[above]       at (P2) {\tiny $\zeta^2 = i$};
        \node[above left]  at (P4) {\tiny $\zeta^4 = -1$};
        \node[below]       at (P6) {\tiny $\zeta^6 = -i$};

        \draw[blue!30, thin] (P1) -- (P5);
        \draw[blue!30, thin] (P3) -- (P7);
    \end{tikzpicture}
    \caption{8-ésimas raízes da unidade}
    \label{fig:raizes_8}
\end{subfigure}

\caption{Comparação entre as raízes da unidade no plano complexo.}
\label{fig:comparacao_raizes}

\end{figure}



De forma mais geral:

\begin{quote}
$ \zeta_N = e^{\frac{2\pi i}{N}} $, uma raiz $N$-ésima primitiva da unidade. Então para todo inteiro $a$, $ {\zeta_N^{a+\frac{N}{2}} = -\,\zeta_N^{a}}. $
\end{quote}
Alem de que, pela periodicidade $\zeta_N^{a+N}=\zeta_N^a$.

Para esse caso a DFT eh representada desse modo:
\[
 \begin{bmatrix} 1 & 1 & 1 & 1 \\ 1 & -i & -1 & i \\ 1 & -1 & 1 & -1 \\ 1 & i & -1 & -i \end{bmatrix}
\]


Voltandao a FFT, o algoritmo decompõe uma DFT de tamanho $N$ em duas sub-transformadas de tamanho $N/2$, separando os índices pares e ímpares da sequência original:

\begin{align}
X[k]
&= \sum_{m=0}^{\frac{N}{2}-1} x[2m] \, \zeta_N^{2mk}
  + \sum_{m=0}^{\frac{N}{2}-1} x[2m+1] \, \zeta_N^{(2m+1)k} \nonumber\\[4pt]
&\text{Usando } \zeta_N^{2}=\zeta_{N/2}:\qquad
\zeta_N^{2mk}=(\zeta_N^{2})^{mk}=\zeta_{N/2}^{mk} \nonumber\\[4pt]
&= \sum_{m=0}^{\frac{N}{2}-1} x[2m] \, \zeta_{N/2}^{mk}
  + \zeta_N^{k} \sum_{m=0}^{\frac{N}{2}-1} x[2m+1] \, \zeta_{N/2}^{mk} \nonumber\\
&= E[k] + \zeta_N^{k} O[k],
\qquad k = 0, \dots, \frac{N}{2}-1.\nonumber
\end{align}

Esta estrutura permite calcular dois valores de saída ($X[k]$ e $X[k+N/2]$) utilizando os mesmos resultados intermediários, através da denominada \textbf{operação borboleta} (\textit{butterfly operation}):

\begin{enumerate}
    \item $X[k] = E[k] + \zeta_N^k O[k]$
    \item $X[k + N/2] = E[k] - \zeta_N^k O[k]$
\end{enumerate}

como pode ser visto na imagem 

\begin{figure}[H]
\includegraphics{images/butterfly4.pdf}
\caption{butterfly radix-2}
\label{fig:butterfly}
\end{figure}
o a implementacao em sage esta no codigo\ref{lst:FFT}

\begin{lstlisting}[caption={Implementação do algoritmo FFT em SageMath},label={lst:FFT}]
def fft(a, omega):
    n = len(a)
    if n == 1:
        return a[:]  

 
    a_par = fft(a[0::2], omega^2)
    a_impar  = fft(a[1::2], omega^2)

    A = [0] * n
    w = 1
    half = n // 2
    for k in range(half):
        t = w * a_impar[k]
        A[k]          = a_par[k] + t
        A[k + half]   = a_par[k] - t
        w *= omega
    return A
\end{lstlisting}

Desse modo, reduzimos a complexidade da transformada de $O(n^2) \text{ para } O(n\cdot \log n)$. Por causa disso, podemos utilizar a FFT, junto com o \textbf{teorema da convolucao}, para multiplicar polinomios em $O(n\cdot \log n)$


\section{ Problemas da FFT}

Uns dos problemas da FTT eh que ela trabalha com ponto flutuante, o que, para computadores, eh um grande problemas que pode causar erro de arredondamentos e, assim causar um falha nos esquemas criptograficos. Alem disso, o polinomio dobram de tamanho a cada concolucao o que rapidamente torna-se um problema tanto computacional quanto de armazenamento .



Solucao: utilizar um transformada que utiliza apenas numeros exatos

\section{ Number Theoretic Transform (NTT)}
\subsection{Fundamentos}

As propriedades que usamos na FFT — em especial a existência de uma raiz $N$-ésima da unidade $\zeta_N$ e o fato de que suas potências percorrem uniformemente o círculo — têm um análogo perfeito em teoria dos números, dentro de corpos (ou anéis) finitos. Isso não é coincidência: a FFT nada mais é do que a transformada de Fourier no grupo cíclico $\mathbb{Z}/N\mathbb{Z}$, e a mesma construção existe em outros contextos algébricos.

Mais formalmente, se $\zeta_N$ é uma raiz $N$-ésima primitiva da unidade, então o conjunto de todas as $N$-ésimas raízes
\[
\mu_N=\{1,\zeta_N,\zeta_N^2,\dots,\zeta_N^{N-1}\}
\]
forma um grupo multiplicativo cíclico de ordem $N$. Existe um isomorfismo natural de grupos
\[
\varphi:\mathbb{Z}/N\mathbb{Z}\to \mu_N,\qquad \varphi([k])=\zeta_N^k,
\]
onde o lado esquerdo usa a soma módulo $N$ e o lado direito usa multiplicação:
\[
\varphi([k+\ell])=\zeta_N^{k+\ell}=\zeta_N^k\zeta_N^\ell=\varphi([k])\,\varphi([\ell]).
\]

\textbf{Raiz primitva}
Diferente da DFT complexa, onde raízes da unidade sempre existem para qualquer $n$, a NTT exige que o corpo finito $\mathbb{Z}_p$ suporte a ordem da transformada. 

Para que exista uma raiz primitiva $n$-ésima da unidade em $\mathbb{Z}_p$, o tamanho da sequência $n$ deve dividir a ordem do grupo multiplicativo do corpo:
\[n \mid (p - 1)\]

Agora, iremos tratar da estrutura aritimetica, na qual a NTT ocorre.

Ela eh definida da seguinte forma:
\[
X[k] = \sum_{n=0}^{N-1}x[n]\omega^{nk} \pmod{p}
\]
ou na forma matricial
\[
\mathbf{X}
=
\mathbf{W}_N \, \mathbf{x},
\qquad
\mathbf{x} =
\begin{bmatrix}
x_0 \\ x_1 \\ \vdots \\ x_{N-1}
\end{bmatrix},
\]

\[
\mathbf{W}_N =
\begin{bmatrix}
1        & 1        & 1        & \cdots & 1 \\
1        & \omega   & \omega^2 & \cdots & \omega^{N-1} \\
1        & \omega^2 & \omega^4 & \cdots & \omega^{2(N-1)} \\
\vdots   & \vdots   & \vdots   & \ddots & \vdots \\
1        & \omega^{N-1} & \omega^{2(N-1)} & \cdots & \omega^{(N-1)(N-1)}
\end{bmatrix},
\]

\[
(\mathbf{W}_N)_{k,n} = \omega^{kn},
\qquad 0 \le k,n \le N-1,
\]

Ela opera no anel quociente
\[
R= \mathbb{Z}_p[x] /(x^n-1)
\]
onde $\mathbb{Z}_p[x]$ representa o coeficientes do polinomio $mod\text{ } p$ e o $/(x^n-1)$ faz com que a cadeia longa de polinomio dobre em si mesma formando um ciclo.

a imagem \ref{fig:torus} ilustra a sequencia das duas operacoes visualmente.

\begin{figure}[H]
    \centering
\includegraphics{images/torus.pdf}
\caption{representação das transformacoes}
\label{fig:torus}
\end{figure}


Escolhemos modulo de um primo para que $Z_p$ seja um field, i.e, para que a aritimedica tenha propiedades agradaveis e escolhemos $(x^n-1)$ para que haja raizes distinta e se preserve as "informacoes" indepentes o que garante que ele possa ser decomposto completamente.

\section{O Isomorfismo via Teorema Chinês dos Restos (CRT)}

A fundamentação algébrica da NTT reside na decomposição do anel de polinômios. Se $\omega$ é uma raiz primitiva $n$-ésima da unidade no corpo finito $\mathbb{Z}_p$, o polinômio $x^n - 1$ pode ser fatorado completamente em binômios lineares distintos:

\[x^n - 1 = \prod_{i=0}^{n-1} (x - \omega^i)\]

Como cada termo $(x - \omega^i)$ é irredutível e todos são coprimos entre si, o \textbf{Teorema Chinês dos Restos (CRT)} garante a existência de um isomorfismo de anéis:

\[\frac{\mathbb{Z}_p[x]}{(x^n - 1)} \cong \frac{\mathbb{Z}_p[x]}{(x - \omega^0)} \times \frac{\mathbb{Z}_p[x]}{(x - \omega^1)} \times \dots \times \frac{\mathbb{Z}_p[x]}{(x - \omega^{n-1})}\]

Este isomorfismo é o que permite interpretar a NTT não apenas como uma transformação de vetores, mas como uma mudança de representação.

A comporação computacional dessa ferramentas pode ser vista na tabela a seguir, onde calculou-se os números de Fibonacci.
\begin{center}
\begin{tabular}{|l|r|}
\hline
Algorithm & Fibonacci index \\ \hline
Algoritmo Naive & 44 \\ \hline
Algoritmo Linear & 566'053 \\ \hline
Algoritmo FFT & 3'145'816 \\ \hline
Algoritmo NTT & 24'178'839 \\ \hline
Algoritmo GMP & 238'961'323 \\ \hline
\end{tabular}
\end{center}

fonte: https://github.com/SheafificationOfG/Fibsonisheaf

\section{Apendix}
\subsection{Exemplo de convolução circular}
\label{sec:convolução}
Sejam dois sinais $x$ e $y$ de comprimento $N=3$:
\[
x = \begin{bmatrix} 1 \\ 2 \\ 0 \end{bmatrix}, \quad y = \begin{bmatrix} 1 \\ 0 \\ 1 \end{bmatrix}
\]

A frequência fundamental é a raiz da unidade $\zeta_3 = e^{-i \frac{2\pi}{3}}$. Usaremos a propriedade fundamental:
\[
1 + \zeta_3 + \zeta_3^2 = 0 \implies \zeta_3 + \zeta_3^2 = -1
\]

\subsection*{Método 1: Convolução no Tempo (Matriz Circulante)}

A convolução circular $z = x \circledast y$ equivale à multiplicação de uma matriz circulante $C_x$ pelo vetor $y$:

\[
C_x = \begin{bmatrix} 
x[0] & x[2] & x[1] \\ 
x[1] & x[0] & x[2] \\ 
x[2] & x[1] & x[0] 
\end{bmatrix} 
= \begin{bmatrix} 
1 & 0 & 2 \\ 
2 & 1 & 0 \\ 
0 & 2 & 1 
\end{bmatrix}
\]

Calculando $z = C_x y$:
\[
z = \begin{bmatrix} 1 & 0 & 2 \\ 2 & 1 & 0 \\ 0 & 2 & 1 \end{bmatrix} 
\begin{bmatrix} 1 \\ 0 \\ 1 \end{bmatrix} 
= \begin{bmatrix} 
1(1) + 0 + 2(1) \\ 
2(1) + 0 + 0 \\ 
0 + 0 + 1(1) 
\end{bmatrix} 
= \begin{bmatrix} 3 \\ 2 \\ 1 \end{bmatrix}
\]

\subsection*{Método 2: Diagonalização (Domínio da Frequência)}

Mostra-se que a mudança de base para a base de Fourier diagonaliza a matriz circulante $C_x$. Matematicamente, isso significa que $C_x$ pode ser decomposta como
$$C_x = F^{-1} \Lambda_x F$$
onde $\Lambda_x$ é uma matriz diagonal contendo os coeficientes da DFT de $x$.

Substituindo essa decomposição na equação original da convolução $z = C_x y$, podemos rearranjar os termos para utilizar a multiplicação na frequência:
\begin{equation}
z = (F^{-1} \Lambda_x F) y = F^{-1} \Lambda_x (F y) = F^{-1} \Lambda_x Y
\end{equation}

Dessa forma, o cálculo se resume a obter $Y$ (DFT de $y$), multiplicar pela matriz diagonal $\Lambda_x$ e aplicar a inversa ($F^{-1}$).

Para legitimar que os autovalores de $C_x$ são os coeficientes da DFT de $x$, resolvemos:
\[
\det(C_x - \lambda I) = 0 \implies 
\det \begin{bmatrix} 
1-\lambda & 0 & 2 \\ 
2 & 1-\lambda & 0 \\ 
0 & 2 & 1-\lambda 
\end{bmatrix} = 0
\]
Expandindo o determinante:
\[
(1-\lambda)^3 + 8 = 0 \implies (1-\lambda)^3 = -8
\]
As raízes para $(1-\lambda)$ são as três raízes cúbicas de $-8$:
\begin{align*}
1 - \lambda_0 = -2 &\implies \lambda_0 = 3 \\
1 - \lambda_1 = -2\zeta_3 &\implies \lambda_1 = 1 + 2\zeta_3 \\
1 - \lambda_2 = -2\zeta_3^2 &\implies \lambda_2 = 1 + 2\zeta_3^2
\end{align*}
Estes valores coincidem exatamente com a DFT de $x$, provando a legitimidade da diagonalização.

\subsection*{Verificação do Autovetor}
Verificamos agora se o autovetor $v_1$ da base de Fourier (coluna de $F^{-1}$), dado por $v_1 = [1, \zeta_3^2, \zeta_3]^T$, satisfaz $C_x v_1 = \lambda_1 v_1$.
\\
Lado esquerdo ($C_x v_1$):
\[
\begin{bmatrix} 1 & 0 & 2 \\ 2 & 1 & 0 \\ 0 & 2 & 1 \end{bmatrix} 
\begin{bmatrix} 1 \\ \zeta_3^2 \\ \zeta_3 \end{bmatrix} 
= 
\begin{bmatrix} 1 + 2\zeta_3 \\ 2 + \zeta_3^2 \\ 2\zeta_3^2 + \zeta_3 \end{bmatrix}
\]
Lado direito ($\lambda_1 v_1$):
\[
(1+2\zeta_3) 
\begin{bmatrix} 1 \\ \zeta_3^2 \\ \zeta_3 \end{bmatrix} 
=
\begin{bmatrix} 
1 + 2\zeta_3 \\ 
\zeta_3^2 + 2\zeta_3^3 \\ 
\zeta_3 + 2\zeta_3^2 
\end{bmatrix} 
=
\begin{bmatrix} 
1 + 2\zeta_3 \\ 
2 + \zeta_3^2 \\ 
\zeta_3 + 2\zeta_3^2 
\end{bmatrix}
\]
A igualdade é satisfeita, pensamento analogo é utilizado para mostrar que o $v_0$ e $v_2$ e  tambem são autovetores, confirmando que a base de Fourier é a base natural de $C_x$.

\subsection*{Representação Matricial da Diagonalização}

Primeiro, calculamos os vetores transformados $X = Fx$ e $Y = Fy$:
\[
X = \begin{bmatrix} 3 \\ 1+2\zeta_3 \\ 1+2\zeta_3^2 \end{bmatrix}, \quad
Y = \begin{bmatrix} 2 \\ 1+\zeta_3^2 \\ 1+\zeta_3 \end{bmatrix}
\]

Agora, construímos a matriz diagonal $\Lambda_x = \text{diag}(X)$. A operação de convolução no domínio da frequência ($Z = X \cdot Y$), visualizada matricialmente como $Z = \Lambda_x Y$, torna-se:

\[
\mathbf{Z} = 
\underbrace{
\begin{bmatrix} 
3 & 0 & 0 \\ 
0 & 1+2\zeta_3 & 0 \\ 
0 & 0 & 1+2\zeta_3^2 
\end{bmatrix}
}_{\text{Matriz Diagonal } (\Lambda_x)}
\begin{bmatrix} 
2 \\ 
1+\zeta_3^2 \\ 
1+\zeta_3 
\end{bmatrix}
\]

Executando o produto matricial (que equivale ao produto ponto a ponto):

\[
\mathbf{Z} = 
\begin{bmatrix} 
3 \cdot 2 \\ 
(1+2\zeta_3)(1+\zeta_3^2) \\ 
(1+2\zeta_3^2)(1+\zeta_3) 
\end{bmatrix}
=
\begin{bmatrix} 
6 \\ 
2 + \zeta_3 \\ 
2 + \zeta_3^2 
\end{bmatrix}
\]
\small{\textit{*Nota: As simplificações algébricas utilizam $1+\zeta_3+\zeta_3^2=0$.}}

\subsection*{Retorno ao Tempo (IDFT)}

Finalmente, aplicamos a matriz inversa de Fourier ($F^{-1}$) para obter $z$:

\[
z = F^{-1} \mathbf{Z} = \frac{1}{3}
\begin{bmatrix}
1 & 1 & 1 \\
1 & \zeta_3^{-1} & \zeta_3^{-2} \\
1 & \zeta_3^{-2} & \zeta_3^{-4}
\end{bmatrix}
\begin{bmatrix} 
6 \\ 
2 + \zeta_3 \\ 
2 + \zeta_3^2 
\end{bmatrix}
=
\begin{bmatrix} 3 \\ 2 \\ 1 \end{bmatrix}
\]

Isso confirma numericamente que a base de Fourier diagonalizou a operação.

\subsection{Determinição unívoca do polinomio}
\label{sec:Determinição unívoca do polinômio}

\begin{theorem}[Unicidade via matriz de Vandermonde]
Sejam $x_0,\dots,x_{N-1}$ escalares dois a dois distintos em um corpo $\mathbb{K}$
(e.g., $\mathbb{R}$, $\mathbb{C}$, $\mathbb{F}_p$), e sejam $y_0,\dots,y_{N-1}\in\mathbb{K}$.
Existe um \textbf{único} polinômio
$p(x)=a_0+a_1x+\cdots+a_{N-1}x^{N-1}\in\mathbb{K}[x]$
tal que $p(x_i)=y_i$ para todo $i=0,\dots,N-1$.
\end{theorem}

\begin{proof}
Escreva
$p(x)=\sum_{k=0}^{N-1} a_k x^k$.
Impor as condições $p(x_i)=y_i$ para $i=0,\dots,N-1$ gera o sistema linear

$$
\begin{bmatrix}
1 & x_0 & x_0^2 & \cdots & x_0^{N-1}\\
1 & x_1 & x_1^2 & \cdots & x_1^{N-1}\\
\vdots & \vdots & \vdots & \ddots & \vdots\\
1 & x_{N-1} & x_{N-1}^2 & \cdots & x_{N-1}^{N-1}
\end{bmatrix}
\begin{bmatrix}
a_0\\
a_1\\
\vdots\\
a_{N-1}
\end{bmatrix}
=
\begin{bmatrix}
y_0\\
y_1\\
\vdots\\
y_{N-1}
\end{bmatrix}.
$$

Denote essa matriz por $V$ (matriz de Vandermonde,i.e. $V_{i,j}=x_i^{j-1}$ para todo os índices i,j), o vetor de coeficientes por $a$
e o vetor de valores por $y$; então o sistema é

$$
Va = y.
$$

O determinante de Vandermonde é dado por

$$
\det(V)=\prod_{0\le i<j\le N-1}(x_j-x_i).
$$

Como os $x_i$ são dois a dois distintos, temos $x_j-x_i\ne 0$ para $i\ne j$,
logo $\det(V)\ne 0$. Portanto, $V$ é invertível e o sistema $Va=y$ tem
\textbf{solução única}, dada por

$$
a = V^{-1}y.
$$

Concluímos que existe um único vetor de coeficientes $(a_0,\dots,a_{N-1})$,
isto é, um \textbf{único} polinômio $p(x)$ de grau $\le N-1$ que interpola os $N$ pontos.
\end{proof}

\end{document}
