\documentclass[12pt]{article}
\usepackage{zocal}
\usepackage{biblatex}
\addbibresource{referencias.bib} % Certifique-se que o nome do arquivo .bib está correto
\begin{document}

\begin{titlepage}
    \begin{center}
        \textbf{\textnormal{Escola Politécnica-USP}}\\
        \textbf{\textnormal{LARC}}

        \vspace{3cm}
        
         \includegraphics[width=0.5\textwidth]{images/Logo-Escola-Politécnica-Minerva.pdf}\\[1cm]
        
        {\Huge Seminário NTT}\\[1.5cm]
        \vspace{2cm}
        
        \textbf{Resumo}\\[0.5cm]
            \begin{minipage}{0.5\textwidth}
            \small
            O objetivo do semanário é apresentar a NTT e suas vantagens, para isso, parti-se da transformada de Fourier, apos, para a sua forma discreta (DFT) e chega-se na NTT propriamente dita.
            O que almejo com esse semanário é passar a intuição por trás das ferramentas matemática utilizadas e beleza inerentes delas.
            \end{minipage}


    \end{center}
\end{titlepage}
\thispagestyle{empty}

% --- SUMÁRIO ---
\tableofcontents
\newpage




\section{ A Dualidade entre Tempo e Frequência}
A Transformada de Fourier é uma operação matemática que mapeia uma função do domínio do tempo (ou espaço) para o seu domínio dual: a frequência. Essa transição é extremamente útil, pois propriedades que são complexas de analisar no tempo tornam-se claras no espectro de frequências.


Esta ferramenta é um pilar fundamental em diversas áreas do conhecimento:

\begin{itemize}
    \item \textbf{Matemática Pura:} Essencial na Teoria Analítica dos Números e no estudo de Equações Diferenciais Parciais (EDPs).
    \item \textbf{Física Moderna:} É a base matemática do \textbf{Princípio da Incerteza de Heisenberg} na Mecânica Quântica, onde a posição e o momento de uma partícula formam um par de variáveis conjugadas de Fourier.
    \item \textbf{Engenharia:} Processamento de sinais, compressão de dados (MP3, JPEG) e telecomunicações.
\end{itemize}


\section{ Transformada de Fourier Contínua (CTFT)}
Para uma função contínua $g(t)$, a transformada é definida pela integral:

\[\mathcal{F}(f) = \int_{-\infty}^{\infty} g(t) e^{-2\pi if t} \, dt\]

Ela pode ser entendida como um produto interno(uma projeção) do sinal com todos as frequencias da reta real $ \langle g, e^{-2\pi i ft} \rangle$, que, devido a ortoganalidade das frequências diferentes e que funcoes bem comportadas podem ser decompostas em series de autofuncoes $e^{-ift}$, consegue extrair extamente as frequências do sinal.
Apesar de sua elegância teórica, a CTFT apresenta desafios para a aplicação prática em sistemas digitais:

\begin{enumerate}
    \item \textbf{Natureza Analítica:} A resolução de integrais impróprias exige uma manipulação simbólica que é difícil de implementarz em computadores comuns.
    \item \textbf{Limite Infinito:} A definição pressupõe que conhecemos o sinal de $-\infty$ a $+\infty$, o que é impossível em cenários reais.
    \item \textbf{Amostragem Finita:} Na prática, os sinais são capturados de forma discreta (amostras) e por um tempo limitado, o que torna a integral contínua inaplicável.
\end{enumerate}


\section{ Transformada Discreta de Fourier (DFT)}
Para viabilizar o processamento em computadores, utilizamos a \textbf{DFT} (\textit{Discrete Fourier Transform}). Ela opera sobre uma sequência finita de $N$ amostras, mapeando dados discretos no tempo para o domínio da frequência. Segundo o Teorema de Amostragem de Nyquist-Shannon \cite{oppenheim2009}, para que um sinal contínuo possa ser unicamente determinado a partir de suas amostras, a frequência de amostragem $f_s$ deve ser superior ao dobro da maior frequência $f_{max}$ contida no sinal. $   f_s > 2 f_{max}$
A DFT é definida da seguinte forma:

\[X[k] = \sum_{n=0}^{N-1} x[n] e^{-i \frac{2\pi}{N} kn}\]

Para $k = 0, 1, \dots, N-1$.

Diferente da versão contínua, a DFT lida com somatórios e vetores numéricos, permitindo que a teoria de Fourier seja aplicada em qualquer dispositivo digital.
é possivel provar que a DFT é um transformacao linear, logo, pode ser representada matricialmente.

Seja $\zeta_N = e^{-i\frac{2\pi}{N}}$. A representação matricial da DFT para $n = 0, 1, \dots, N-1$ é:

\[
\begin{bmatrix} 
X[0] \\ X[1] \\ \vdots \\ X[N-1] 
\end{bmatrix}
=
\begin{bmatrix}
1 & 1 & 1 & \dots & 1 \\
1 & \zeta_N^{1} & \zeta_N^{2} & \dots & \zeta_N^{N-1} \\
1 & \zeta_N^{2} & \zeta_N^{4} & \dots & \zeta_N^{2(N-1)} \\
\vdots & \vdots & \vdots & \ddots & \vdots \\
1 & \zeta_N^{N-1} & \zeta_N^{2(N-1)} & \dots & \zeta_N^{(N-1)(N-1)}
\end{bmatrix}
\begin{bmatrix} 
x[0] \\ x[1] \\ \vdots \\ x[N-1] 
\end{bmatrix}
\]

\subsection{Definição da IDFT}
A reconstrução do sinal original no domínio do tempo a partir de suas amostras de frequência é realizada pela IDFT:

\[
x[n] = \frac{1}{N} \sum_{k=0}^{N-1} X[k] \zeta_N^{-nk}, \quad n = 0, 1, \dots, N-1
\]

Onde $\zeta_N^{-nk} = e^{i \frac{2\pi}{N} nk}$. Matricialmente, a IDFT é dada por:

\[
\begin{bmatrix} 
x[0] \\ x[1] \\ x[2] \\ \vdots \\ x[N-1] 
\end{bmatrix}
=
\frac{1}{N}
\begin{bmatrix}
1 & 1 & 1 & \dots & 1 \\
1 & \zeta_N^{-1} & \zeta_N^{-2} & \dots & \zeta_N^{-(N-1)} \\
1 & \zeta_N^{-2} & \zeta_N^{-4} & \dots & \zeta_N^{-2(N-1)} \\
\vdots & \vdots & \vdots & \ddots & \vdots \\
1 & \zeta_N^{-(N-1)} & \zeta_N^{-2(N-1)} & \dots & \zeta_N^{-(N-1)(N-1)}
\end{bmatrix}
\begin{bmatrix} 
X[0] \\ X[1] \\ X[2] \\ \vdots \\ X[N-1] 
\end{bmatrix}
\]
Percebe-se que nesse caso discreto, a DFT e IDFT atuam como uma matriz mudanca de base, saindo da base do tempo e indo para base das raizes unitarias

A DFT pde ser vista de outra óptica, como avalição de polinomios.
Seja
\[
a(x)=\sum_{n=0}^{N-1} a_n x^n
\]
e tome $\zeta_N=e^{-\frac{2\pi i}{N}}$ seja uma $N$-ésima raiz primitiva da unidade. Define
\[
A_k := a(\zeta_N^k).
\]
Então 
\[
A_k
= \sum_{n=0}^{N-1} a_n(\zeta_N^k)^n
= \sum_{n=0}^{N-1} a_n\,\zeta_N^{kn},
\]
o que é exatamente a formulação da DFT, para $a_n \equiv x[n]$ e $A_n \equiv X[n]$, opostamente, a IDFT é interpretada como interpolção de polinômios:

Dados os valores
\(\{A_k\}_{k=0}^{N-1}\), ela reconstrói os coeficientes \(\{a_n\}_{n=0}^{N-1}\) do
único polinômio de grau \(N-1\) que satisfaz \(a(\zeta_N^k)=A_k\) para todo \(k\).
Explicitamente,
\[
a_n=\frac{1}{N}\sum_{k=0}^{N-1} A_k\,\zeta_N^{-kn}.
\]
Assim, \(\mathrm{DFT}\) é \emph{avaliar} em raízes da unidade e \(\mathrm{IDFT}\) é
\emph{interpolar} (recuperar os coeficientes) a partir dessas avaliações. (Essa perspectiva baseia-se na determinição unívoca do polinómio de grau $N-1$ por N pontos veja a seção~\ref{sec:Determinição unívoca do polinômio} para demonstração desse fato)


\section{ A Multiplicação de Polinômios e a Complexidade Computacional}

Um problema simplificado pela mudanca de domino é a multiplicacao de polinomios. Tome os polinomios $f(x)$ e $g(x)$ de grau $N-1$: 

\[f(x) = \sum_{i=0}^{N-1} a_i x^i, \quad g(x) = \sum_{j=0}^{N-1} b_j x^j\]

Na abordagem clássica , o produto $h(x) = f(x) \cdot g(x)$ é obtido distribuindo-se cada termo de $f$ sobre todos os termos de $g$. Este processo resulta em um novo polinômio de grau $2N-2$:

\[h(x) = \sum_{k=0}^{2N-2} c_k x^k\]
onde $c_k = \sum_{i+j=k} a_i b_j$.

Nesta metodologia, o cálculo de cada coeficiente $c_k$ exige múltiplas operações de produto e soma, resultando em uma complexidade assintótica $O(n^2)$. Para polinômios com grandes volumes de coeficientes, este custo computacional torna o metodo inviavel.



A conexão entre a multiplicação de polinômios e a análise de Fourier vem do fato de que,
se $h(x)=f(x)g(x)$, então os coeficientes $c_k$ de $h$ são dados pela \textbf{convolução linear}
dos coeficientes de $f$ e $g$:
\[
c_k=\sum_{i+j=k} a_i\, b_{j}.
\]

Neste trabalho, devido ao foco na NTT e ao anel quociente $\mathbb{Z}_p[x]/(x^N+1)$,
trabalhamos com a \textbf{convolução circular} (de comprimento $n$), definida por
\[
c_k=\sum_{i=0}^{N-1} a_i\, b_{(k-i)\bmod n}, \qquad k=0,\dots,N-1.
\]

\textbf{Teorema da Convolução:} a transformada de uma convolução no domínio do tempo (ou espaço)
é o produto ponto a ponto (Hadamard) das transformadas no domínio da frequência:
\[
\mathcal{F}(f*g)=\mathcal{F}(f)\odot \mathcal{F}(g).
\]

Assim, o cálculo custoso da convolução  é convertido em um produto ponto a ponto,
pois a base de Fourier diagonaliza o operador de convolução (circulante).
Com a transformada direta ingênua o custo ainda é $O(n^2)$, não há ganho da multiplicação clássica de polinômios.

Uma exemplo para o convencimento do leitor foi disposto no apendice~\ref{sec:convolução}
\section{ A Fast Fourier Transform}

A FFT (Fast Fourier Transform) é uma maneira de otimizar o calculo da DFT.

O algoritmo da FFT foi redescoberta por Cooley e Tukey em 1965, uma vez que Gauss ja tinha utilzado um algoritmo semelhante para calcular a orbitas de asteroides em 1805. 

O algoritmo se baseia em \textbf{dividir para conquistar}.

\begin{quote}
Relembrando
\end{quote}

As raízes unitárias possuem propriedades cíclicas e certas simetrias que permitem a economia nos cálculos, vejamos um exemplo.
\[\zeta_4^1 = e^{-i\frac{2\pi}{4}} = e^{-i90^\circ} = -i\]

\[
\begin{aligned}
\zeta^1 &= -i \qquad\qquad
\zeta^2 = -1 \\[6pt]
\zeta^3 &=  i \qquad\qquad
\zeta^4 = 1
\end{aligned}
\]
por isso, percebe-se que, a cada 2 "deslocamentos", o valor se torna o oposto, como ilustrado na figura:

%verificar se usar com duas raizes ou apenas a 4-esima raiz
%raizes
\begin{figure}[H]
\centering

% --- Subfigura 1: 4-ésimas raízes ---
\begin{subfigure}[b]{0.48\linewidth}
    \centering
    \begin{tikzpicture}[scale=1.8]
        \def\R{1}
        \draw[gray, thin, dashed] (0,0) circle (\R);
        \draw[->, >=Stealth, gray!50] (-1.2,0) -- (1.2,0) node[right] {\tiny Re};
        \draw[->, >=Stealth, gray!50] (0,-1.2) -- (0,1.2) node[above] {\tiny Im};

        \foreach \k in {0,1,2,3} {
            \coordinate (P\k) at ({\k*360/4}:\R);
            \fill (P\k) circle (1.5pt);
        }

        \node[above right] at (P0) {\tiny $1$};
        \node[above right] at (P1) {\tiny $\zeta = i$};
        \node[above left]  at (P2) {\tiny $\zeta^2 = -1$};
        \node[below left]  at (P3) {\tiny $\zeta^3 = -i$};

        \draw[blue!30, thin] (P0) -- (P2);
        \draw[blue!30, thin] (P1) -- (P3);
    \end{tikzpicture}
    \caption{4-ésimas raízes da unidade}
    \label{fig:raizes_4}
\end{subfigure}
\hfill
% --- Subfigura 2: 8-ésimas raízes ---
\begin{subfigure}[b]{0.48\linewidth}
    \centering
    \begin{tikzpicture}[scale=1.8]
        \def\R{1}
        \draw[gray, thin, dashed] (0,0) circle (\R);
        \draw[->, >=Stealth, gray!50] (-1.2,0) -- (1.2,0) node[right] {\tiny Re};
        \draw[->, >=Stealth, gray!50] (0,-1.2) -- (0,1.2) node[above] {\tiny Im};

        \foreach \k in {0,...,7} {
            \coordinate (P\k) at ({\k*360/8}:\R);
            \fill (P\k) circle (1.5pt);
        }

        \node[above right] at (P0) {\tiny $1$};
        \node[above right] at (P1) {\tiny $\zeta$};
        \node[above]       at (P2) {\tiny $\zeta^2 = i$};
        \node[above left]  at (P4) {\tiny $\zeta^4 = -1$};
        \node[below]       at (P6) {\tiny $\zeta^6 = -i$};

        \draw[blue!30, thin] (P1) -- (P5);
        \draw[blue!30, thin] (P3) -- (P7);
    \end{tikzpicture}
    \caption{8-ésimas raízes da unidade}
    \label{fig:raizes_8}
\end{subfigure}

\caption{Comparação entre as raízes da unidade no plano complexo.}
\label{fig:comparacao_raizes}

\end{figure}



De forma mais geral:

\begin{quote}
$ \zeta_N = e^{\frac{-2\pi i}{N}} $, uma raiz $N$-ésima primitiva da unidade. Então para todo inteiro $a$, $ {\zeta_N^{a+\frac{N}{2}} = -\,\zeta_N^{a}}. $
\end{quote}
Alem de que, pela periodicidade $\zeta_N^{a+N}=\zeta_N^a$.

Para esse caso a DFT é representada desse modo:
\[
 \begin{bmatrix} 1 & 1 & 1 & 1 \\ 1 & -i & -1 & i \\ 1 & -1 & 1 & -1 \\ 1 & i & -1 & -i \end{bmatrix}
\]


Voltandao a FFT, o algoritmo decompõe uma DFT de tamanho $N$ em duas sub-transformadas de tamanho $N/2$, separando os índices pares e ímpares da sequência original:

\begin{align}
X[k]
&= \sum_{m=0}^{\frac{N}{2}-1} x[2m] \, \zeta_N^{2mk}
  + \sum_{m=0}^{\frac{N}{2}-1} x[2m+1] \, \zeta_N^{(2m+1)k} \nonumber\\[4pt]
&\text{Usando } \zeta_N^{2}=\zeta_{N/2}:\qquad
\zeta_N^{2mk}=(\zeta_N^{2})^{mk}=\zeta_{N/2}^{mk} \nonumber\\[4pt]
&= \sum_{m=0}^{\frac{N}{2}-1} x[2m] \, \zeta_{N/2}^{mk}
  + \zeta_N^{k} \sum_{m=0}^{\frac{N}{2}-1} x[2m+1] \, \zeta_{N/2}^{mk} \nonumber\\
&= E[k] + \zeta_N^{k} O[k],
\qquad k = 0, \dots, \frac{N}{2}-1.\nonumber
\end{align}

Esta estrutura permite calcular dois valores de saída ($X[k]$ e $X[k+N/2]$) utilizando os mesmos resultados intermediários, através da denominada \textbf{operação borboleta} (\textit{butterfly operation}):

\begin{enumerate}
    \item $X[k] = E[k] + \zeta_N^k O[k]$
    \item $X[k + N/2] = E[k] - \zeta_N^k O[k]$
\end{enumerate}

como pode ser visto na imagem 

\begin{figure}[H]
\includegraphics{images/butterfly4.pdf}
\caption{butterfly radix-2}
\label{fig:butterfly}
\end{figure}
o a implementacao em sage esta no codigo\ref{lst:FFT}

\begin{lstlisting}[caption={Implementação do algoritmo FFT em SageMath},label={lst:FFT}]
def fft(a, omega):
    n = len(a)
    if n == 1:
        return a[:]  

 
    a_par = fft(a[0::2], omega^2)
    a_impar  = fft(a[1::2], omega^2)

    A = [0] * n
    w = 1
    half = n // 2
    for k in range(half):
        t = w * a_impar[k]
        A[k]          = a_par[k] + t
        A[k + half]   = a_par[k] - t
        w *= omega
    return A
\end{lstlisting}

Desse modo, reduzimos a complexidade da transformada de $O(n^2) \text{ para } O(n\cdot \log n)$. Por causa disso, podemos utilizar a FFT, junto com o \textbf{teorema da convolucao}, para multiplicar polinomios em $O(n\cdot \log n)$


\section{ Problemas da FFT}

Uns dos problemas da FTT é que ela trabalha com ponto flutuante, o que, para computadores, é um grande problemas que pode causar erro de arredondamentos e, assim causar uma falha nos esquemas criptograficos. Alem disso, na convolução linear, o polinomio dobra de tamanho a cada concolucao o que rapidamente torna-se um problema tanto computacional quanto de armazenamento .



Solucao: utilizar um transformada que utiliza apenas numeros exatos

\section{ Number Theoretic Transform (NTT)}
\subsection{Fundamentos}

As propriedades que usamos na FFT — em especial a existência de uma raiz $N$-ésima da unidade $\zeta_N$ e o fato de que suas potências percorrem uniformemente o círculo — têm um análogo perfeito em teoria dos números, dentro de corpos (ou anéis) finitos. Isso não é coincidência: a FFT nada mais é do que a transformada de Fourier no grupo cíclico $\mathbb{Z}/N\mathbb{Z}$, e a mesma construção existe em outros contextos algébricos.

Mais formalmente, se $\zeta_N$ é uma raiz $N$-ésima primitiva da unidade, então o conjunto de todas as $N$-ésimas raízes
\[
\mu_N=\{1,\zeta_N,\zeta_N^2,\dots,\zeta_N^{N-1}\}
\]
forma um grupo multiplicativo cíclico de ordem $N$. Existe um isomorfismo natural de grupos
\[
\varphi:\mathbb{Z}/N\mathbb{Z}\to \mu_N,\qquad \varphi([k])=\zeta_N^k,
\]
onde o lado esquerdo usa a soma módulo $N$ e o lado direito usa multiplicação:
\[
\varphi([k+\ell])=\zeta_N^{k+\ell}=\zeta_N^k\zeta_N^\ell=\varphi([k])\,\varphi([\ell]).
\]

\subsection*{Raiz Primitiva e Estrutura Negacíclica}
Diferente da DFT complexa, onde raízes da unidade sempre existem para qualquer $N$, a NTT exige que o corpo finito $\mathbb{Z}_p$, isto é, os inteiros$\pmod p$, sendo $p$ um primo, suporte a ordem da transformada.

Para a NTT Negacíclica, utilizada para realizar a multiplicação polinomial módulo $x^N + 1$, precisamos de uma raiz primitiva $2N$-ésima da unidade em $\mathbb{Z}_p$, que denotaremos por $\psi$. Isso implica que:
\begin{enumerate}
    \item $\psi^{2N} \equiv 1 \pmod{p}$;
    \item $\psi^N \equiv -1 \pmod{p}$.
\end{enumerate}

Para garantir a existência desse elemento, a ordem $2N$ deve dividir a ordem do grupo multiplicativo do corpo. Portanto, o primo $p$ deve satisfazer:
\[ 2N \mid (p - 1) \qquad \text{i.e. 2N divide (p-1)}\]
Ou seja, $p \equiv 1 \pmod{2N}$.

Como $2N \mid (p-1)$, em particular temos $p \nmid N$ (i.e. $gcd(p,N)=1$); logo $N$ possui inverso em $\mathbb{Z}_p$. 



Agora, trataremos da estrutura algébrica. A NTT Negacíclica é definida no anel quaociente 
\[
R = \frac{\mathbb{Z}_p[x]}{(x^N+1)}
\]

(a imagem \ref{fig:torus} ilustra a sequencia das duas operacoes visualmente.)

 Ela pode ser vista como a avalição do polinômio nas raízes da equação $x^N + 1 = 0$, que correspondem às potências ímpares de $\psi$. A transformada é definida por:

\[
X[k] = \sum_{n=0}^{N-1} x[n] \psi^{(2k+1)n} \pmod{p}
\]

Na forma matricial:

\[
\mathbf{X} = \mathbf{W}_N \, \mathbf{x}, \qquad \mathbf{x} = 
\begin{bmatrix} 
x_0 \\ x_1 \\ \vdots \\ x_{N-1} 
\end{bmatrix}
\]

Neste caso, a matriz de transformação $\mathbf{W}_N$ difere da versão cíclica padrão, pois seus coeficientes seguem a estrutura das raízes negacíclicas:

\[
\mathbf{W}_N = 
\begin{bmatrix}
1 & 1 & 1 & \cdots & 1 \\
1 & \psi^{3} & \psi^{6} & \cdots & \psi^{3(N-1)} \\
1 & \psi^{5} & \psi^{10} & \cdots & \psi^{5(N-1)} \\
\vdots & \vdots & \vdots & \ddots & \vdots \\
1 & \psi^{(2k+1)} & \psi^{(2k+1)2} & \cdots & \psi^{(2k+1)(N-1)} \\
\vdots & \vdots & \vdots & \ddots & \vdots \\
1 & \psi^{(2N-1)} & \psi^{(2N-1)2} & \cdots & \psi^{(2N-1)(N-1)}
\end{bmatrix}
\]

De forma geral, o termo na linha $k$ e coluna $n$ da matriz é dado por:
\[
(\mathbf{W}_N)_{k,n} = \psi^{(2k+1)n} \pmod{p}, \qquad 0 \le k,n \le N-1
\]


\begin{figure}[H]
    \centering
\includegraphics{images/torus.pdf}
\caption{representação das transformacoes}
\label{fig:torus}
\end{figure}


\section{O Isomorfismo via Teorema Chinês dos Restos (CRT)}

A fundamentação algébrica da NTT reside na estrutura do anel quociente $\mathbb{Z}_p[x] / \langle x^N + 1 \rangle$. Dado que $x^N + 1$ fatora-se em $N$ binômios lineares distintos $(x - \psi^{2k+1})$, o CRT estabelece uma equivalência entre o anel original e o produto de anéis menores.

Para compreender a natureza desses anéis menores, invocamos o conceito de \textit{homomorfismo  }. Considere o anel quociente genérico $\mathbb{Z}_p[x] / \langle x - \alpha \rangle$. Pela Divisão Euclidiana, qualquer polinômio $a(x) \in \mathbb{Z}_p[x]$ pode ser escrito unicamente como:
\[
a(x) = q(x) \cdot (x - \alpha) + r
\]
onde o grau de $r$ deve ser estritamente menor que o grau do divisor $(x - \alpha)$. Como o divisor tem grau 1, o resto $r$ é necessariamente um escalar constante ($r \in \mathbb{Z}_p$).

No anel quociente, impomos que $x - \alpha \equiv 0$, ou seja, $x \equiv \alpha$. Ao aplicarmos essa equivalência (ou avaliarmos a equação em $x = \alpha$), o termo que contém o divisor se anula:
\[
a(\alpha) = q(\alpha) \cdot (\underbrace{\alpha - \alpha}_{0}) + r \implies a(\alpha) = r
\]
Portanto, a classe de equivalência de $a(x)$ módulo $(x - \alpha)$ é univocamente representada pelo escalar $a(\alpha)$. Isso estabelece o isomorfismo local:
\[
\frac{\mathbb{Z}_p[x]}{\langle x - \alpha \rangle} \cong \mathbb{Z}_p \quad \text{via o mapa} \quad [a(x)] \mapsto a(\alpha)
\]

Com essa intuição firmada, podemos expandir explicitamente o isomorfismo global $\Phi$ garantido pelo CRT. A transformação mapeia o polinômio $a(x)$ diretamente para o vetor de suas avaliações nas raízes de $x^N + 1$:

\[
\begin{aligned}
\Phi : \frac{\mathbb{Z}_p[x]}{\langle x^N + 1 \rangle} &\xrightarrow{\cong} \bigotimes_{k=0}^{N-1} \frac{\mathbb{Z}_p[x]}{\langle x - \psi^{2k+1} \rangle} \\
&\cong \underbrace{\mathbb{Z}_p \times \mathbb{Z}_p \times \dots \times \mathbb{Z}_p}_{N \text{ vezes}}
\end{aligned}
\]

A ação de $\Phi$ sobre um elemento $a(x)$ é dada explicitamente por:
\[
\Phi(a(x)) = \left( a(\psi^1), a(\psi^3), \dots, a(\psi^{2N-1}) \right) \in \mathbb{Z}_p^N
\]

Neste domínio transformado, a operação de multiplicação de dois polinômios $a(x)$ e $b(x)$ ocorre componente a componente. Se $C = \Phi(a(x) \cdot b(x))$, então a $k$-ésima componente do vetor resultante é:
\[
C_k = (a \cdot b)(\psi^{2k+1}) = a(\psi^{2k+1}) \cdot b(\psi^{2k+1})
\]
Isso demonstra formalmente por que a convolução cíclica (ou negacíclica) no domínio do tempo se traduz em um produto de Hadamard (ponto a ponto) no domínio da frequência.

\begin{figure}[H]
\centering
\resizebox{\textwidth}{!}{%
  \includegraphics{images/CRT_NTT.pdf}

}
\caption{CRT da NTT Negacíclica}
\label{fig:NTT}
\end{figure}

\section{Transformada Numérica Inversa (INTT)}

Como o Teorema Chinês dos Restos garante que a aplicação da NTT é um isomorfismo bijetor entre o anel de polinômios $\mathbb{Z}_p[x]/(x^N+1)$ e o domínio da frequência, existe uma transformação inversa única capaz de recuperar os coeficientes originais.

Denotamos a inversa multiplicativa de $N$ no corpo $\mathbb{Z}_p$ por $N^{-1}$, tal que $N \cdot N^{-1} \equiv 1 \pmod{p}$. A **NTT Negacíclica Inversa (INTT)** é definida formalmente por:

\[
x[n] = N^{-1} \sum_{k=0}^{N-1} X[k] \psi^{-(2k+1)n} \pmod{p}
\]

Note que o termo $\psi^{-(2k+1)n}$ refere-se à potência do inverso multiplicativo da raiz, ou seja, $\psi^{-1} \equiv \psi^{2N-1} \pmod{p}$.

\subsection*{Representação Matricial}
Na forma matricial, a operação de inversão corresponde à resolução do sistema linear $\mathbf{X} = \mathbf{W}_N \mathbf{x}$. A solução é dada por:

\[
\mathbf{x} = \mathbf{W}_N^{-1} \mathbf{X}
\]

Onde a matriz inversa $\mathbf{W}_N^{-1}$ é definida como a conjugada transposta da matriz direta, escalada pelo fator $N^{-1}$:

\[
\mathbf{W}_N^{-1} = N^{-1}
\begin{bmatrix}
1 & 1 & 1 & \cdots & 1 \\
1 & \psi^{-3} & \psi^{-6} & \cdots & \psi^{-(2N-1)} \\
1 & \psi^{-6} & \psi^{-10} & \cdots & \psi^{-2(2N-1)} \\
\vdots & \vdots & \vdots & \ddots & \vdots \\
1 & \psi^{-(2k+1)} & \psi^{-2(2k+1)} & \cdots & \psi^{-(N-1)(2k+1)} \\
\vdots & \vdots & \vdots & \ddots & \vdots \\
1 & \psi^{-(2N-1)} & \psi^{-2(2N-1)} & \cdots & \psi^{-(N-1)(2N-1)}
\end{bmatrix}
\]

De modo geral, o termo na linha $n$ e coluna $k$ da matriz inversa é:
\[
(\mathbf{W}_N^{-1})_{n,k} = N^{-1} \psi^{-(2k+1)n} \pmod{p}
\]

\subsection*{Ortogonalidade}
A existência da inversa e a validade da fórmula acima decorrem da propriedade de ortogonalidade das raízes negacíclicas no corpo finito $\mathbb{Z}_p$. Para quaisquer $n, m \in \{0, \dots, N-1\}$, temos:

\[
\sum_{k=0}^{N-1} \psi^{(2k+1)(n-m)} = 
\begin{cases} 
N & \text{se } n = m \\
0 & \text{se } n \neq m 
\end{cases}
\]

Isso garante que o produto das matrizes resulta na identidade:
\[
\mathbf{W}_N^{-1} \mathbf{W}_N = \mathbf{I}
\]

\subsection{A NTT como uma FFT: Decomposição Radix-2 via CRT}

A eficiência da NTT reside na estratégia ``dividir para conquistar''. Em vez de avaliar um polinômio $A(x)$ de grau $N-1$ de uma só vez, utilizamos a estrutura algébrica do anel para decompor o problema em dois problemas independentes de tamanho $N/2$.

\subsubsection*{1. A Fatoração do Módulo}
O ponto de partida é a decomposição do polinômio $x^N + 1$.
Como $\psi$ é uma raiz primitiva $2N$-ésima da unidade, temos que $\psi^N \equiv -1 \pmod p$, o que implica $(\psi^{N/2})^2 \equiv -1$.

Pelo Teorema Chinês dos Restos (CRT), podemos fatorar o módulo original em dois sub-anéis:
\[
x^N + 1 = x^N - (\psi^{N/2})^2 = \underbrace{(x^{N/2} - \psi^{N/2})}_{\text{Módulo Esquerdo}} \cdot \underbrace{(x^{N/2} + \psi^{N/2})}_{\text{Módulo Direito}}
\]

\textit{Nota: Essa fatoração particiona o conjunto das raízes de $x^N+1$ (que são as potências ímpares $\psi^{2k+1}$). As raízes que satisfazem o lado esquerdo continuam na recursão $L$, e as que satisfazem o lado direito seguem na recursão $R$.}

\subsubsection*{2. O Colapso da Variável}
Considere o polinômio de entrada $A(x)$ com coeficientes $a[i]$. Podemos dividir seu somatório em duas partes: os primeiros $N/2$ termos e os últimos $N/2$ termos.

\[
A(x) = \sum_{i=0}^{N/2-1} a[i]x^i + \sum_{i=0}^{N/2-1} a[i+N/2]x^{i+N/2}
\]

Colocando $x^{N/2}$ em evidência na segunda metade, obtemos uma estrutura que revela a simetria do problema:
\[
A(x) = \sum_{i=0}^{N/2-1} \left( a[i] + x^{N/2} \cdot a[i+N/2] \right) x^i
\]

A intuição chave da NTT é projetar essa equação nos sub-anéis definidos no passo 1. Ao fazer isso, a potência $x^{N/2}$ deixa de ser uma variável e torna-se uma constante escalar:

\begin{itemize}
    \item \textbf{Ramo Esquerdo} ($x^{N/2} \equiv \psi^{N/2}$): Substituímos $x^{N/2}$ por $+\psi^{N/2}$.
    \item \textbf{Ramo Direito} ($x^{N/2} \equiv -\psi^{N/2}$): Substituímos $x^{N/2}$ por $-\psi^{N/2}$.
\end{itemize}

\subsubsection*{3. A Operação Borboleta (Butterfly)}
Ao realizar as substituições acima diretamente no somatório, os coeficientes dos novos polinômios reduzidos ($a_L$ e $a_R$) surgem imediatamente. Para cada posição $0 \le i < N/2$:

\[
\boxed{
\begin{aligned}
    a_L[i] &= a[i] + \psi^{N/2} \cdot a[i+N/2] \pmod p \\
    a_R[i] &= a[i] - \psi^{N/2} \cdot a[i+N/2] \pmod p
\end{aligned}
}
\]

Note que não precisamos calcular polinômios inteiros; a operação acontece pontualmente nos coeficientes. Os valores $a_L$ tornam-se a entrada para a recursão da esquerda, e $a_R$ para a recursão da direita.

\begin{figure}[H]
    \centering
    \includegraphics{images/butterflyNTT.pdf}
    \caption{Butterfly NTT}
    \label{fig:bntt}
\end{figure}


\begin{figure}[H]
    \centering
    \includegraphics{images/butterflyNTT_4.pdf}
    \caption{Butterfly NTT}
    \label{fig:b4ntt}
\end{figure}

\subsection{A INTT via Gentleman-Sande (GS): Reconstrução via CRT}

Enquanto a transformação direta (NTT) utiliza a decomposição para reduzir o problema, a transformação inversa (INTT) realiza o caminho oposto: ela combina os resultados dos sub-problemas menores para reconstruir o vetor original. O algoritmo de Gentleman-Sande implementa essa reconstrução de forma eficiente.

\subsubsection*{1. A Inversão do Sistema Linear}
Na etapa da transformação direta, definimos os valores dos sub-ramos ($a_L$ e $a_R$) através das equações:
\[
\begin{cases}
    a_L[i] = a[i] + \psi^{N/2} \cdot a[i+N/2] \\
    a_R[i] = a[i] - \psi^{N/2} \cdot a[i+N/2]
\end{cases}
\]
Na etapa inversa, conhecemos $a_L$ e $a_R$ (retornados pela recursão) e nosso objetivo é recuperar os coeficientes originais $a[i]$ e $a[i+N/2]$. Trata-se apenas de resolver este sistema linear de duas variáveis.

\subsubsection*{2. A Lógica da Recuperação}
Somando e subtraindo as duas equações do sistema acima, podemos isolar os termos desejados:

\begin{itemize}
    \item \textbf{Recuperando a metade inferior ($a[i]$):}
    Somamos as equações:
    \[
    a_L[i] + a_R[i] = 2a[i] \implies a[i] = \frac{a_L[i] + a_R[i]}{2}
    \]
    
    \item \textbf{Recuperando a metade superior ($a[i+N/2]$):}
    Subtraímos a segunda da primeira:
    \[
    a_L[i] - a_R[i] = 2\psi^{N/2} \cdot a[i+N/2]
    \]
    Para isolar $a[i+N/2]$, multiplicamos pelo inverso de $2\psi^{N/2}$:
    \[
    a[i+N/2] = \frac{a_L[i] - a_R[i]}{2} \cdot \psi^{-N/2}
    \]
\end{itemize}

Esta manipulação algébrica revela a diferença fundamental na ordem das operações entre a ida (Cooley-Tukey) e a volta (Gentleman-Sande).

\subsubsection*{3. A Borboleta GS (Inverse Butterfly)}
Baseado nas isolações acima, definimos a operação borboleta inversa. Para cada par de entradas $a_L[i]$ e $a_R[i]$ provenientes dos sub-blocos, calculamos:

\[
\boxed{
\begin{aligned}
    a[i] &= \frac{1}{2} (a_L[i] + a_R[i]) \pmod p \\
    a[i+N/2] &= \frac{1}{2} (a_L[i] - a_R[i]) \cdot \psi^{-N/2} \pmod p
\end{aligned}
}
\]

Observe a mudança estrutural:
\begin{enumerate}
    \item Na \textbf{Ida (CT)}, multiplicamos pelo fator de rotação \textit{antes} de somar/subtrair.
    \item Na \textbf{Volta (GS)}, subtraímos primeiro e multiplicamos pelo fator de rotação inverso \textit{depois}.
\end{enumerate}

\textit{Nota de Implementação: O fator escalar $1/2$ (inverso modular de 2) geralmente não é aplicado a cada camada. Para eficiência, acumula-se o fator total $1/N$ e aplica-se uma única multiplicação escalar ao final de toda a transformação.}

\subsubsection*{4. Fluxo de Dados e Bit-Reversal}
O algoritmo Gentleman-Sande é a "transposta" do Cooley-Tukey.
Se a NTT direta recebe os dados em ordem natural e os devolve em ordem \textit{bit-reversed} (permutada), a INTT via GS aceita os dados em ordem \textit{bit-reversed} e os reconstrói naturalmente para a ordem linear.

Isso cria um par perfeito:
\[ \text{Input Natural} \xrightarrow{\text{CT}} \text{Bit-Reversed} \xrightarrow{\text{Operação Pontual}} \text{Bit-Reversed} \xrightarrow{\text{GS}} \text{Output Natural} \]
Essa simetria elimina a necessidade de reordenamentos de memória (bit-reversal) custosos entre as transformações.

\begin{figure}[H]
    \centering
    \includegraphics{images/INTT_butterfly.pdf}
    \caption{Butterfly INTT}
    \label{fig:bINNT}
\end{figure}
A implementa da NTT negacíclica ésta a seguir \ref{lst:NTT}

\begin{lstlisting}[caption=codigo em SageMath,label={lst:NTT}]
def NTT_Negacyclic(A, R, psi):
    
    n = len(A)
    if n == 1:
        return [A[0]]

    A_even = A[0::2]
    A_odd  = A[1::2]

    psi_sq = psi^2
    Y_even = NTT_Negacyclic(A_even, R, psi_sq)
    Y_odd  = NTT_Negacyclic(A_odd,  R, psi_sq)

    Y = [R(0)] * n
    
    w = psi 
    
    half = n // 2
    for k in range(half):
        t = w * Y_odd[k]
        
        Y[k]        = Y_even[k] + t
        Y[k + half] = Y_even[k] - t
        
        w *= psi_sq 

    return Y

def INTT_Negacyclic(A, R, psi):
    n = len(A)
    # A inversa usa psi^-1
    psi_inv = psi^(-1)
    
    Y = NTT_Negacyclic(A, R, psi_inv)
    
    n_inv = R(n)^(-1)
    return [y * n_inv for y in Y]
\end{lstlisting}

A comporação computacional dessa ferramentas pode ser vista na tabela a seguir, onde calculou-se os números de Fibonacci.
\begin{center}
\begin{tabular}{|l|r|}
\hline
Algorithm & Fibonacci index \\ \hline
Algoritmo Naive & 44 \\ \hline
Algoritmo Linear & 566'053 \\ \hline
Algoritmo FFT & 3'145'816 \\ \hline
Algoritmo NTT & 24'178'839 \\ \hline
Algoritmo GMP & 238'961'323 \\ \hline
\end{tabular}
\end{center}

fonte: https://github.com/SheafificationOfG/Fibsonisheaf

\subsection{Extra: NTT incompleta e a restrição de parâmetros no caso negacíclico}

Na multiplicação negacíclica de polinômios no anel
\[
\mathbb{Z}_p[x]/(x^N + 1),
\]
com $N$ potência de dois, a utilização de uma \textbf{NTT completa} exige a existência
de uma raiz primitiva de ordem $2N$ módulo $p$. Quando $p$ é primo, essa condição é
equivalente a
\[
p \equiv 1 \pmod{2N}.
\]
Em aplicações criptográficas baseadas em reticulados, em especial em esquemas de Fully Homomorphic Encryption
o valor de $N$ é tipicamente muito grande. Como consequência, a condição acima impõe
fortes restrições sobre a escolha do primo $p$, frequentemente forçando o uso de
módulos grandes ou pouco flexíveis, o que impacta tanto a eficiência quanto o ajuste
fino de parâmetros de segurança.

Uma forma natural de relaxar essa restrição é empregar a chamada \textbf{NTT
$\ell$-incompleta}. A ideia consiste em interromper o algoritmo da NTT antes de
executar todos os $\log_2 N$ estágios do esquema radix-2. Mais precisamente, ao parar
após $\log_2 N - \ell$ estágios, a existência da transformada passa a requerer apenas
uma raiz da unidade de ordem $2N/2^{\ell}$ em $\mathbb{Z}_p$, o que resulta na condição
mais fraca
\[
p \equiv 1 \pmod{2N/2^{\ell}}.
\]
Dessa forma, o conjunto de primos admissíveis torna-se significativamente maior,
permitindo escolhas de parâmetros mais flexíveis.

Do ponto de vista algébrico, a NTT $\ell$-incompleta não avalia o polinômio em todas as
$N$ raízes da unidade, mas o mapeia para um vetor de $N/2^{\ell}$ polinômios de menor
grau, cada um pertencente a um quociente do tipo
\[
\mathbb{Z}_p[x]/(x^{2^{\ell}} - \psi_i),
\]
onde $\psi_i$ são potências apropriadas da raiz disponível. A multiplicação passa então
a ser realizada componente a componente nesses anéis menores, seguida de uma
transformada inversa incompleta.

O principal custo adicional desse método está na \emph{multiplicação de base} dentro
dos anéis $\mathbb{Z}_p[x]/(x^{2^{\ell}} - \psi_i)$, que, para valores maiores de
$\ell$, tende a ser implementada por algoritmos quadráticos. No entanto, o trabalho
de \textcite{paiva_incomplete_ntt}  mostra que, no contexto de bootstrapping amortizado para esquemas
do tipo FHEW/TFHE, esse custo pode sera eliminado. Os autores reformulam a NTT inversa
como o produto de duas matrizes de borboleta (controladas por um parâmetro de divisão
$\alpha$) e demonstram que a multiplicação de base da NTT incompleta pode ser
\emph{incorporada} ao primeiro estágio dessa decomposição matricial, sem aumento no
número assintótico de operações.

Como resultado, obtém-se o que os autores denominam \emph{``NTT incompleta gratuita''}:
para valores moderados de $\ell$, é possível relaxar substancialmente a condição sobre
o primo $p$ sem penalidade computacional relevante, ao mesmo tempo em que se ampliam
as opções de parâmetros e se melhora o compromisso entre desempenho e taxa de falha
(decryption failure rate).

\section{Agradecimentos}
Gostaria de agradecer ao professor doutor \emph{Thales Paiva} pela chance de apresentar o semanário e pelos ensinamentos,
gostaria de agradecer tambem o professor doutor \emph{João Fernando da Cunha Nariyoshi} por esclarecer algumas dúvidas sobre a parte da matemática pura.
Foi realmente divertido pesquisar, escrever, desenhar e animar para esse semanário.

\section{Apêndice}
\subsection{Exemplo de convolução circular}
\label{sec:convolução}
Sejam dois sinais $x$ e $y$ de comprimento $N=3$:
\[
x = \begin{bmatrix} 1 \\ 2 \\ 0 \end{bmatrix}, \quad y = \begin{bmatrix} 1 \\ 0 \\ 1 \end{bmatrix}
\]

A frequência fundamental é a raiz da unidade $\zeta_3 = e^{-i \frac{2\pi}{3}}$. Usaremos a propriedade fundamental:
\[
1 + \zeta_3 + \zeta_3^2 = 0 \implies \zeta_3 + \zeta_3^2 = -1
\]

\subsection*{Método 1: Convolução no Tempo (Matriz Circulante)}

A convolução circular $z = x \circledast y$ equivale à multiplicação de uma matriz circulante $C_x$ pelo vetor $y$:

\[
C_x = \begin{bmatrix} 
x[0] & x[2] & x[1] \\ 
x[1] & x[0] & x[2] \\ 
x[2] & x[1] & x[0] 
\end{bmatrix} 
= \begin{bmatrix} 
1 & 0 & 2 \\ 
2 & 1 & 0 \\ 
0 & 2 & 1 
\end{bmatrix}
\]

Calculando $z = C_x y$:
\[
z = \begin{bmatrix} 1 & 0 & 2 \\ 2 & 1 & 0 \\ 0 & 2 & 1 \end{bmatrix} 
\begin{bmatrix} 1 \\ 0 \\ 1 \end{bmatrix} 
= \begin{bmatrix} 
1(1) + 0 + 2(1) \\ 
2(1) + 0 + 0 \\ 
0 + 0 + 1(1) 
\end{bmatrix} 
= \begin{bmatrix} 3 \\ 2 \\ 1 \end{bmatrix}
\]

\subsection*{Método 2: Diagonalização (Domínio da Frequência)}

Mostra-se que a mudança de base para a base de Fourier diagonaliza a matriz circulante $C_x$. Matematicamente, isso significa que $C_x$ pode ser decomposta como
$$C_x = F^{-1} \Lambda_x F$$
onde $\Lambda_x$ é uma matriz diagonal contendo os coeficientes da DFT de $x$.

Substituindo essa decomposição na equação original da convolução $z = C_x y$, podemos rearranjar os termos para utilizar a multiplicação na frequência:
\begin{equation}
z = (F^{-1} \Lambda_x F) y = F^{-1} \Lambda_x (F y) = F^{-1} \Lambda_x Y
\end{equation}

Dessa forma, o cálculo se resume a obter $Y$ (DFT de $y$), multiplicar pela matriz diagonal $\Lambda_x$ e aplicar a inversa ($F^{-1}$).

Para legitimar que os autovalores de $C_x$ são os coeficientes da DFT de $x$, resolvemos:
\[
\det(C_x - \lambda I) = 0 \implies 
\det \begin{bmatrix} 
1-\lambda & 0 & 2 \\ 
2 & 1-\lambda & 0 \\ 
0 & 2 & 1-\lambda 
\end{bmatrix} = 0
\]
Expandindo o determinante:
\[
(1-\lambda)^3 + 8 = 0 \implies (1-\lambda)^3 = -8
\]
As raízes para $(1-\lambda)$ são as três raízes cúbicas de $-8$:
\begin{align*}
1 - \lambda_0 = -2 &\implies \lambda_0 = 3 \\
1 - \lambda_1 = -2\zeta_3 &\implies \lambda_1 = 1 + 2\zeta_3 \\
1 - \lambda_2 = -2\zeta_3^2 &\implies \lambda_2 = 1 + 2\zeta_3^2
\end{align*}
Estes valores coincidem exatamente com a DFT de $x$, provando a legitimidade da diagonalização.

\subsection*{Verificação do Autovetor}
Verificamos agora se o autovetor $v_1$ da base de Fourier (coluna de $F^{-1}$), dado por $v_1 = [1, \zeta_3^2, \zeta_3]^T$, satisfaz $C_x v_1 = \lambda_1 v_1$.
\\
Lado esquerdo ($C_x v_1$):
\[
\begin{bmatrix} 1 & 0 & 2 \\ 2 & 1 & 0 \\ 0 & 2 & 1 \end{bmatrix} 
\begin{bmatrix} 1 \\ \zeta_3^2 \\ \zeta_3 \end{bmatrix} 
= 
\begin{bmatrix} 1 + 2\zeta_3 \\ 2 + \zeta_3^2 \\ 2\zeta_3^2 + \zeta_3 \end{bmatrix}
\]
Lado direito ($\lambda_1 v_1$):
\[
(1+2\zeta_3) 
\begin{bmatrix} 1 \\ \zeta_3^2 \\ \zeta_3 \end{bmatrix} 
=
\begin{bmatrix} 
1 + 2\zeta_3 \\ 
\zeta_3^2 + 2\zeta_3^3 \\ 
\zeta_3 + 2\zeta_3^2 
\end{bmatrix} 
=
\begin{bmatrix} 
1 + 2\zeta_3 \\ 
2 + \zeta_3^2 \\ 
\zeta_3 + 2\zeta_3^2 
\end{bmatrix}
\]
A igualdade é satisfeita, pensamento analogo é utilizado para mostrar que o $v_0$ e $v_2$ e  tambem são autovetores, confirmando que a base de Fourier é a base natural de $C_x$.

\subsection*{Representação Matricial da Diagonalização}

Primeiro, calculamos os vetores transformados $X = Fx$ e $Y = Fy$:
\[
X = \begin{bmatrix} 3 \\ 1+2\zeta_3 \\ 1+2\zeta_3^2 \end{bmatrix}, \quad
Y = \begin{bmatrix} 2 \\ 1+\zeta_3^2 \\ 1+\zeta_3 \end{bmatrix}
\]

Agora, construímos a matriz diagonal $\Lambda_x = \text{diag}(X)$. A operação de convolução no domínio da frequência ($Z = X \cdot Y$), visualizada matricialmente como $Z = \Lambda_x Y$, torna-se:

\[
\mathbf{Z} = 
\underbrace{
\begin{bmatrix} 
3 & 0 & 0 \\ 
0 & 1+2\zeta_3 & 0 \\ 
0 & 0 & 1+2\zeta_3^2 
\end{bmatrix}
}_{\text{Matriz Diagonal } (\Lambda_x)}
\begin{bmatrix} 
2 \\ 
1+\zeta_3^2 \\ 
1+\zeta_3 
\end{bmatrix}
\]

Executando o produto matricial (que equivale ao produto ponto a ponto):

\[
\mathbf{Z} = 
\begin{bmatrix} 
3 \cdot 2 \\ 
(1+2\zeta_3)(1+\zeta_3^2) \\ 
(1+2\zeta_3^2)(1+\zeta_3) 
\end{bmatrix}
=
\begin{bmatrix} 
6 \\ 
2 + \zeta_3 \\ 
2 + \zeta_3^2 
\end{bmatrix}
\]
\small{\textit{*Nota: As simplificações algébricas utilizam $1+\zeta_3+\zeta_3^2=0$.}}

\subsection*{Retorno ao Tempo (IDFT)}

Finalmente, aplicamos a matriz inversa de Fourier ($F^{-1}$) para obter $z$:

\[
z = F^{-1} \mathbf{Z} = \frac{1}{3}
\begin{bmatrix}
1 & 1 & 1 \\
1 & \zeta_3^{-1} & \zeta_3^{-2} \\
1 & \zeta_3^{-2} & \zeta_3^{-4}
\end{bmatrix}
\begin{bmatrix} 
6 \\ 
2 + \zeta_3 \\ 
2 + \zeta_3^2 
\end{bmatrix}
=
\begin{bmatrix} 3 \\ 2 \\ 1 \end{bmatrix}
\]

Isso confirma numericamente que a base de Fourier diagonalizou a operação.

\subsection{Determinição unívoca do polinomio}
\label{sec:Determinição unívoca do polinômio}

\begin{theorem}[Unicidade via matriz de Vandermonde]
Sejam $x_0,\dots,x_{N-1}$ escalares dois a dois distintos em um corpo $\mathbb{K}$
(e.g., $\mathbb{R}$, $\mathbb{C}$, $\mathbb{F}_p$), e sejam $y_0,\dots,y_{N-1}\in\mathbb{K}$.
Existe um \textbf{único} polinômio
$p(x)=a_0+a_1x+\cdots+a_{N-1}x^{N-1}\in\mathbb{K}[x]$
tal que $p(x_i)=y_i$ para todo $i=0,\dots,N-1$.
\end{theorem}

\begin{proof}
Escreva
$p(x)=\sum_{k=0}^{N-1} a_k x^k$.
Impor as condições $p(x_i)=y_i$ para $i=0,\dots,N-1$ gera o sistema linear

$$
\begin{bmatrix}
1 & x_0 & x_0^2 & \cdots & x_0^{N-1}\\
1 & x_1 & x_1^2 & \cdots & x_1^{N-1}\\
\vdots & \vdots & \vdots & \ddots & \vdots\\
1 & x_{N-1} & x_{N-1}^2 & \cdots & x_{N-1}^{N-1}
\end{bmatrix}
\begin{bmatrix}
a_0\\
a_1\\
\vdots\\
a_{N-1}
\end{bmatrix}
=
\begin{bmatrix}
y_0\\
y_1\\
\vdots\\
y_{N-1}
\end{bmatrix}.
$$

Denote essa matriz por $V$ (matriz de Vandermonde,i.e. $V_{i,j}=x_i^{j-1}$ para todo os índices i,j), o vetor de coeficientes por $a$
e o vetor de valores por $y$; então o sistema é

$$
Va = y.
$$

O determinante de Vandermonde é dado por

$$
\det(V)=\prod_{0\le i<j\le N-1}(x_j-x_i).
$$

Como os $x_i$ são dois a dois distintos, temos $x_j-x_i\ne 0$ para $i\ne j$,
logo $\det(V)\ne 0$. Portanto, $V$ é invertível e o sistema $Va=y$ tem
\textbf{solução única}, dada por

$$
a = V^{-1}y.
$$

Concluímos que existe um único vetor de coeficientes $(a_0,\dots,a_{N-1})$,
isto é, um \textbf{único} polinômio $p(x)$ de grau $\le N-1$ que interpola os $N$ pontos.
\end{proof}

\end{document}
\bibliographystyle{plain}
\bibliography{referencias}
