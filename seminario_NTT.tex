\documentclass[12pt]{article}
\usepackage{zocal}
\usepackage{booktabs}
\usepackage{biblatex}
\newtheorem{teorema}{Teorema}
\newtheorem{proposicao}{Proposição}
\renewcommand{\proofname}{Demonstração}
\renewcommand{\algorithmicrequire}{\textbf{Input:}}
\renewcommand{\algorithmicensure}{\textbf{Output:}}

\addbibresource{referencias.bib} %
\begin{document}
\nocite{*}
\begin{titlepage}
    \begin{center}
        \textbf{\textnormal{Escola Politécnica-USP}}\\
        \textbf{\textnormal{LARC}}

        \vspace{3cm}
        
         \includegraphics[width=0.5\textwidth]{images/Logo-Escola-Politécnica-Minerva.pdf}\\[1cm]
        
        {\Huge \texttt{Seminário NTT}}\\[1.5cm]
        \vspace{2cm}
        
        \textbf{Resumo}\\[0.5cm]
            \begin{minipage}{0.5\textwidth}
            \small\texttt{
            O objetivo do seminário é apresentar a NTT e suas vantagens, para isso, parte-se da transformada de Fourier, em seguida, para a sua forma discreta (DFT) e chega-se na NTT propriamente dita.
            O que almejo com esse trabalho é passar a intuição por trás das ferramentas matemática utilizadas e beleza inerentes a elas.
            }
            \end{minipage}


    \end{center}
\end{titlepage}
\thispagestyle{empty}

% --- SUMÁRIO ---
\tableofcontents
\newpage




\section{ A Dualidade entre Tempo e Frequência}
A Transformada de Fourier é uma operação matemática que mapeia uma função do domínio do tempo (ou espaço) para o seu domínio dual: a frequência. Essa transição é extremamente útil, pois propriedades que são complexas de analisar no tempo tornam-se claras no espectro de frequências.


Esta ferramenta é um pilar fundamental em diversas áreas do conhecimento:

\begin{itemize}
    \item \textbf{Matemática Pura:} Essencial na Teoria Analítica dos Números e no estudo de Equações Diferenciais Parciais (EDPs).
    \item \textbf{Física Moderna:} É a base matemática do \textbf{Princípio da Incerteza de Heisenberg} na Mecânica Quântica, onde a posição e o momento de uma partícula formam um par de variáveis conjugadas de Fourier.
    \item \textbf{Engenharia:} Processamento de sinais, compressão de dados (MP3, JPEG) e telecomunicações.
\end{itemize}


\section{ Transformada de Fourier Contínua (CFT)}
Para uma função contínua $g(t)$, a transformada é definida pela integral:

\[\mathcal{F}(f) = \int_{-\infty}^{\infty} g(t) e^{-2\pi if t} \, dt\]

Ela pode ser entendida como um produto interno (uma projeção) do sinal com todas as frequências da reta real $ \langle g, e^{-2\pi i ft} \rangle$, que, devido à ortogonalidade das frequências diferentes e que funções bem comportadas podem ser decompostas em séries de autofunções $e^{-2\pi ift}$, consegue extrair exatamente as frequências do sinal.
Apesar de sua elegância teórica, a CFT apresenta desafios para a aplicação prática em sistemas digitais:

\begin{enumerate}
    \item \textbf{Natureza Analítica:} A resolução de integrais impróprias exige uma manipulação simbólica que é difícil de implementar em computadores comuns.
    \item \textbf{Limite Infinito:} A definição pressupõe que conhecemos o sinal de $-\infty$ a $+\infty$, o que é impossível em cenários reais.
    \item \textbf{Amostragem Finita:} Na prática, os sinais são capturados de forma discreta (amostras) e por um tempo limitado, o que torna a integral contínua inaplicável.
\end{enumerate}


\section{ Transformada Discreta de Fourier (DFT)}
Para viabilizar o processamento em computadores, utilizamos a \textbf{DFT} (\textit{Discrete Fourier Transform}). Ela opera sobre uma sequência finita de $N$ amostras, mapeando dados discretos no tempo para o domínio da frequência. Segundo o Teorema de Amostragem de Nyquist-Shannon \cite{oppenheim2009}, para que um sinal contínuo possa ser unicamente determinado a partir de suas amostras, a frequência de amostragem $f_s$ deve ser superior ao dobro da maior frequência $f_{max}$ contida no sinal. $   f_s > 2 f_{max}$
. Com essa restrição para a quantidade de amostras N em mente, a DFT é definida da seguinte forma:

\[X[k] = \sum_{n=0}^{N-1} x[n] e^{-i \frac{2\pi}{N} kn}\]

Para $k = 0, 1, \dots, N-1$.

Diferente da versão contínua, a DFT lida com somatórios e vetores numéricos, permitindo que a teoria de Fourier seja aplicada em qualquer dispositivo digital.
É possível provar que a DFT é uma transformação linear \ref{sec:linear}, logo, pode ser representada matricialmente.

Seja $\zeta_N = e^{-i\frac{2\pi}{N}}$. A representação matricial da DFT para $n = 0, 1, \dots, N-1$ é:

\[
\begin{bmatrix} 
X[0] \\ X[1] \\ \vdots \\ X[N-1] 
\end{bmatrix}
=
\begin{bmatrix}
1 & 1 & 1 & \dots & 1 \\
1 & \zeta_N^{1} & \zeta_N^{2} & \dots & \zeta_N^{N-1} \\
1 & \zeta_N^{2} & \zeta_N^{4} & \dots & \zeta_N^{2(N-1)} \\
\vdots & \vdots & \vdots & \ddots & \vdots \\
1 & \zeta_N^{N-1} & \zeta_N^{2(N-1)} & \dots & \zeta_N^{(N-1)(N-1)}
\end{bmatrix}
\begin{bmatrix} 
x[0] \\ x[1] \\ \vdots \\ x[N-1] 
\end{bmatrix}
\]

\subsection{Definição da IDFT}
A reconstrução do sinal original no domínio do tempo a partir de suas amostras de frequência é realizada pela IDFT:

\[
x[n] = \frac{1}{N} \sum_{k=0}^{N-1} X[k] \zeta_N^{-nk}, \quad n = 0, 1, \dots, N-1
\]

Onde $\zeta_N^{-nk} = e^{i \frac{2\pi}{N} nk}$. Matricialmente, a IDFT é dada por:

\[
\begin{bmatrix} 
x[0] \\ x[1] \\ x[2] \\ \vdots \\ x[N-1] 
\end{bmatrix}
=
\frac{1}{N}
\begin{bmatrix}
1 & 1 & 1 & \dots & 1 \\
1 & \zeta_N^{-1} & \zeta_N^{-2} & \dots & \zeta_N^{-(N-1)} \\
1 & \zeta_N^{-2} & \zeta_N^{-4} & \dots & \zeta_N^{-2(N-1)} \\
\vdots & \vdots & \vdots & \ddots & \vdots \\
1 & \zeta_N^{-(N-1)} & \zeta_N^{-2(N-1)} & \dots & \zeta_N^{-(N-1)(N-1)}
\end{bmatrix}
\begin{bmatrix} 
X[0] \\ X[1] \\ X[2] \\ \vdots \\ X[N-1] 
\end{bmatrix}
\]
Percebe-se que nesse caso discreto, a DFT e IDFT atuam como uma matriz mudança de base, saindo da base do tempo e indo para base das raízes unitárias

Além disso, a DFT pode ser vista de outra óptica, como avaliação de polinômios.

Seja:
\[
a(x)=\sum_{n=0}^{N-1} a_n x^n
\]
e  $\zeta_N=e^{-\frac{2\pi i}{N}}$ Seja uma $N$-ésima raiz primitiva da unidade. Define
\[
A_k := a(\zeta_N^k).
\]
Então 
\[
A_k
= \sum_{n=0}^{N-1} a_n(\zeta_N^k)^n
= \sum_{n=0}^{N-1} a_n\,\zeta_N^{kn},
\]
o que é exatamente a formulação da DFT, para $a_n \equiv x[n]$ e $A_n \equiv X[n]$, opostamente, a IDFT é interpretada como interpolação de polinômios:

Dados os valores
\(\{A_k\}_{k=0}^{N-1}\), ela reconstrói os coeficientes \(\{a_n\}_{n=0}^{N-1}\) do
único polinômio de grau \(N-1\) que satisfaz \(a(\zeta_N^k)=A_k\) para todo \(k\).
Explicitamente,
\[
a_n=\frac{1}{N}\sum_{k=0}^{N-1} A_k\,\zeta_N^{-kn}.
\]
Assim, \(\mathrm{DFT}\) é \emph{avaliar} em raízes da unidade e \(\mathrm{IDFT}\) é
\emph{interpolar} (recuperar os coeficientes) a partir dessas avaliações. (Essa perspectiva baseia-se na determinação unívoca do polinômio de grau $N-1$ por N pontos veja a seção~\ref{sec:Determinição unívoca do polinômio} para demonstração desse fato)



\section{ A Multiplicação de Polinômios e a Complexidade Computacional}

Um problema simplificado pela mudança de domínio é a multiplicação de polinômios. Tome os polinômios $f(x)$ e $g(x)$ de grau $N-1$: 

\[f(x) = \sum_{i=0}^{N-1} a_i x^i, \quad g(x) = \sum_{j=0}^{N-1} b_j x^j\]

Na abordagem clássica, o produto $h(x) = f(x) \cdot g(x)$ é obtido distribuindo-se cada termo de $f$ sobre todos os termos de $g$. Este processo resulta em um novo polinômio de grau $2N-2$:

\[h(x) = \sum_{k=0}^{2N-2} c_k x^k\]
onde $c_k = \sum_{i+j=k} a_i b_j$.

Nesta metodologia, o cálculo de cada coeficiente $c_k$ exige múltiplas operações de produto e soma, resultando em uma complexidade assintótica $O(n^2)$. Para polinômios com grandes volumes de coeficientes, este custo computacional torna o método inviável.



A conexão entre a multiplicação de polinômios e a análise de Fourier vem do fato de que,
se $h(x)=f(x)g(x)$, então os coeficientes $c_k$ de $h$ são dados pela \textbf{convolução linear}
dos coeficientes de $f$ e $g$:
\[
c_k=\sum_{i+j=k} a_i\, b_{j}.
\]

Neste trabalho, devido ao foco na NTT e ao anel quociente $\mathbb{Z}_p[x]/(x^N+1)$,
trabalhamos com a \textbf{convolução negacíclica} (de comprimento $N$), definida por:
\[
c_k = \sum_{i=0}^{k} a_i\, b_{k-i} - \sum_{i=k+1}^{N-1} a_i\, b_{N+k-i}, \qquad k=0,\dots,N-1.
\]
Diferente da circular, aqui os termos que ``ultrapassam'' o comprimento $N$ retornam com sinal invertido.

\subsection{Teorema da Convolução}
\begin{teorema}
  A transformada de uma convolução no domínio do tempo (ou espaço)
é o produto ponto a ponto (Hadamard) das transformadas no domínio da frequência:
\[
\mathcal{F}(f *_{\text{neg}} g) = \mathcal{F}(f) \odot \mathcal{F}(g).
\]

\end{teorema}
Assim, o cálculo custoso da convolução é convertido em um produto ponto a ponto,
pois a base de Fourier apropriada (utilizando as raízes de $x^N+1$) diagonaliza o operador de convolução negacíclica (matriz negacirculante).
Com a transformada direta ingênua, o custo ainda seria $O(N^2)$, portanto, não há ganho computacional em realizar essa transformação.

Um exemplo para o convencimento do leitor foi disposto no apêndice~\ref{sec:convolução}.
\section{ A "Fast Fourier Transform"}

A FFT (Fast Fourier Transform) é uma maneira de otimizar o cálculo da DFT.

O algoritmo da FFT foi redescoberto por Cooley e Tukey em 1965, uma vez que Gauss já tinha utilizado um algoritmo semelhante para calcular as órbitas de asteroides em 1805. 

O algoritmo se baseia em \textbf{dividir para conquistar}.

\hrulefill
\subsection{Números complexos}

As raízes unitárias possuem propriedades cíclicas e certas simetrias que permitem a economia nos cálculos, vejamos um exemplo.
\[\zeta_4^1 = e^{-i\frac{2\pi}{4}} = e^{-i90^\circ} = -i\]

\[
\begin{aligned}
\zeta^1 &= -i \qquad\qquad
\zeta^2 = -1 \\[6pt]
\zeta^3 &=  i \qquad\qquad
\zeta^4 = 1
\end{aligned}
\]
por isso, percebe-se que, a cada 2 "deslocamentos", o valor se torna o oposto, como ilustrado na figura\ref{fig:comparacao_raizes}:

%verificar se usar com duas raízes ou apenas a 4-esima raiz
%raízes
\begin{figure}[H]
\centering

% --- Subfigura 1: 4-ésimas raízes ---
\begin{subfigure}[b]{0.48\linewidth}
    \centering
    \begin{tikzpicture}[scale=1.8]
        \def\R{1}
        \draw[gray, thin, dashed] (0,0) circle (\R);
        \draw[->, >=Stealth, gray!50] (-1.2,0) -- (1.2,0) node[right] {\tiny Re};
        \draw[->, >=Stealth, gray!50] (0,-1.2) -- (0,1.2) node[above] {\tiny Im};

        \foreach \k in {0,1,2,3} {
            \coordinate (P\k) at ({\k*360/4}:\R);
            \fill (P\k) circle (1.5pt);
        }

        \node[above right] at (P0) {\tiny $1$};
        \node[above right] at (P1) {\tiny $\zeta = i$};
        \node[above left]  at (P2) {\tiny $\zeta^2 = -1$};
        \node[below left]  at (P3) {\tiny $\zeta^3 = -i$};

        \draw[blue!30, thin] (P0) -- (P2);
        \draw[blue!30, thin] (P1) -- (P3);
    \end{tikzpicture}
    \caption{4-ésimas raízes da unidade}
    \label{fig:raizes_4}
\end{subfigure}
\hfill
% --- Subfigura 2: 8-ésimas raízes ---
\begin{subfigure}[b]{0.48\linewidth}
    \centering
    \begin{tikzpicture}[scale=1.8]
        \def\R{1}
        \draw[gray, thin, dashed] (0,0) circle (\R);
        \draw[->, >=Stealth, gray!50] (-1.2,0) -- (1.2,0) node[right] {\tiny Re};
        \draw[->, >=Stealth, gray!50] (0,-1.2) -- (0,1.2) node[above] {\tiny Im};

        \foreach \k in {0,...,7} {
            \coordinate (P\k) at ({\k*360/8}:\R);
            \fill (P\k) circle (1.5pt);
        }

        \node[above right] at (P0) {\tiny $1$};
        \node[above right] at (P1) {\tiny $\zeta$};
        \node[above]       at (P2) {\tiny $\zeta^2 = i$};
        \node[above left]  at (P4) {\tiny $\zeta^4 = -1$};
        \node[below]       at (P6) {\tiny $\zeta^6 = -i$};

        \draw[blue!30, thin] (P1) -- (P5);
        \draw[blue!30, thin] (P3) -- (P7);
    \end{tikzpicture}
    \caption{8-ésimas raízes da unidade}
    \label{fig:raizes_8}
\end{subfigure}

\caption{Comparação entre as raízes da unidade no plano complexo.}
\label{fig:comparacao_raizes}

\end{figure}



De forma mais geral:

\begin{quote}
$ \zeta_N = e^{\frac{-2\pi i}{N}} $, uma raiz $N$-ésima primitiva da unidade. Então para todo inteiro $a$, $ {\zeta_N^{a+\frac{N}{2}} = -\,\zeta_N^{a}}. $
\end{quote}
Além de que, pela periodicidade $\zeta_N^{a+N}=\zeta_N^a$ e $\zeta_N^{2}=\zeta_{N/2}$, esta última pode ser vista na figura \ref{fig:comparacao_raizes}, onde $\zeta_8^2=\zeta_4$ 

Para $N=4$, a DFT é representada desse modo:
\[
 \begin{bmatrix} 1 & 1 & 1 & 1 \\ 1 & -i & -1 & i \\ 1 & -1 & 1 & -1 \\ 1 & i & -1 & -i \end{bmatrix}
\]

\hrulefill

Voltando à FFT, o algoritmo decompõe uma DFT de tamanho $N$ em duas sub-transformadas de tamanho $N/2$, separando os índices pares e ímpares da sequência original:

\begin{align}
X[k]
&= \sum_{m=0}^{\frac{N}{2}-1} x[2m] \, \zeta_N^{2mk}
  + \sum_{m=0}^{\frac{N}{2}-1} x[2m+1] \, \zeta_N^{(2m+1)k} \nonumber\\[4pt]
&\text{Usando } \zeta_N^{2}=\zeta_{N/2}:\qquad
\zeta_N^{2mk}=(\zeta_N^{2})^{mk}=\zeta_{N/2}^{mk} \nonumber\\[4pt]
&= \sum_{m=0}^{\frac{N}{2}-1} x[2m] \, \zeta_{N/2}^{mk}
  + \zeta_N^{k} \sum_{m=0}^{\frac{N}{2}-1} x[2m+1] \, \zeta_{N/2}^{mk} \nonumber\\
&= E[k] + \zeta_N^{k} O[k],
\qquad k = 0, \dots, \frac{N}{2}-1.\nonumber
\end{align}

Esta estrutura permite calcular dois valores de saída ($X[k]$ e $X[k+N/2]$) utilizando os mesmos resultados intermediários, através da denominada \textbf{operação borboleta} (\textit{butterfly operation}):

\begin{align}
    X[k] &= E[k] + \zeta_N^k O[k] \\
    X[k + N/2] &= E[k] - \zeta_N^k O[k]
\end{align}

como pode ser visto na imagem \ref{fig:butterfly}

\begin{figure}[H]
\includegraphics{images/butterfly4.pdf}
\caption{butterfly radix-2}
\label{fig:butterfly}
\end{figure}

A implementação em pseudocódigo no algoritmo \ref{alg:fft}

\begin{algorithm}
\caption{Cooley--Tukey FFT}
\label{alg:fft}
\begin{algorithmic}[1]
\Require $x \in \mathbb{C}^N$, $N = 2^m$
\Ensure $X = \mathrm{DFT}(x)$

\If{$N = 1$}
    \State \Return $x$
\EndIf

\State $x_e \gets$ elementos pares de $x$
\State $x_o \gets$ elementos ímpares de $x$

\State $X_e \gets \text{FFT}(x_e)$
\State $X_o \gets \text{FFT}(x_o)$

\For{$k = 0$ to $N/2 - 1$}
    \State $X[k] \gets X_e[k] + \zeta_N^k X_o[k]$
    \State $X[k+N/2] \gets X_e[k] - \zeta_N^k X_o[k]$
\EndFor
\end{algorithmic}
\end{algorithm}

Como a IDFT é praticamente idêntica à DFT, podemos usar o mesmo algoritmo da FFT para calcular a IFFT
\begin{algorithm}
\caption{IFFT via FFT}
\label{alg:ifft-via-fft}
\begin{algorithmic}[1]
\Require $X \in \mathbb{C}^N$
\Ensure $x = \mathrm{IFFT}(X)$

\State $Y \gets \overline{X}$ \Comment{i.e. o conjugado de X}
\State $Z \gets \mathrm{FFT}(Y)$
\State $x \gets \dfrac{1}{N} \cdot \overline{Z}$

\State \Return $x$
\end{algorithmic}
\end{algorithm}


Desse modo, reduzimos a complexidade da transformada de $O(n^2) \longrightarrow O(n\cdot \log n)$. Por causa disso, podemos utilizar a FFT, junto com o \textbf{Teorema da convolução}, para multiplicar polinômios em $O(n\cdot \log n)$ 


\section{ Problemas da FFT}

Um dos problemas da FFT é que ela trabalha com ponto flutuante, o que, para computadores, é um grande problema que pode causar erros de arredondamento ou limitação nos cálculos. Além disso, na convolução linear, o polinômio dobra de tamanho a cada convolução, o que rapidamente torna-se um problema tanto computacional quanto de armazenamento.



Solução: utilizar uma transformada que utiliza apenas números exatos e realiza uma convolução (nega)cíclica

\section{ Number Theoretic Transform (NTT)}

\subsection{Fundamentos}
\hrulefill

As propriedades que usamos na FFT — em especial a existência de uma raiz $N$-ésima da unidade $\zeta_N$ e o fato de que suas potências percorrem uniformemente o círculo — têm um análogo perfeito em teoria dos números, dentro de corpos (ou anéis) finitos. Isso não é coincidência: a DFT nada mais é do que a transformada de Fourier no grupo cíclico $\mathbb{Z}/N\mathbb{Z}$, e a mesma construção existe em outros contextos algébricos.

Mais formalmente, se $\zeta_N$ é uma raiz $N$-ésima primitiva da unidade, então o conjunto de todas as $N$-ésimas raízes
\[
\mu_N=\{1,\zeta_N,\zeta_N^2,\dots,\zeta_N^{N-1}\}
\]
forma um grupo multiplicativo cíclico de ordem $N$. Existe um isomorfismo natural de grupos
\[
\varphi:\mathbb{Z}/N\mathbb{Z}\to \mu_N,\qquad \varphi([k])=\zeta_N^k,
\]
onde o lado esquerdo usa a soma módulo $N$ e o lado direito usa multiplicação:
\[
\varphi([k+\ell])=\zeta_N^{k+\ell}=\zeta_N^k\zeta_N^\ell=\varphi([k])\,\varphi([\ell]).
\]
Assim, o grupo $\mu_N$, gerado pelas potências de $\zeta_N$, é uma realização ``multiplicativa'' do grupo cíclico $\mathbb{Z}/N\mathbb{Z}$: o isomorfismo $[k]\mapsto \zeta_N^k$ mostra que somar módulo $N$ corresponde exatamente a multiplicar raízes $N$-ésimas da unidade. Portanto, estamos apenas vendo o mesmo grupo abstrato em duas roupagens diferentes, uma aditiva e outra multiplicativa.



\subsubsection{Raiz Primitiva e Estrutura Negacíclica}
Diferente da DFT complexa, onde raízes da unidade sempre existem para qualquer $N$, a NTT exige que o corpo finito $\mathbb{Z}_p$, isto é, os inteiros$\pmod p$, sendo $p$ um primo, suporte a ordem da transformada.


Como $p$ é primo, o conjunto $\mathbb{Z}/p\mathbb{Z}$ forma um corpo, e portanto seus elementos não nulos formam um grupo multiplicativo
\[
(\mathbb{Z}/p\mathbb{Z})^\times
\]
de cardinalidade $p-1$. Além disso, esse grupo é cíclico (ver Apêndice~\ref{ap:fp-ciclico}), isto é, existe um gerador $g$ tal que todo elemento não nulo pode ser escrito como uma potência de $g$.

Nesse contexto, dizer que existe uma raiz $2N$-ésima primitiva da unidade em $\mathbb{Z}/p\mathbb{Z}$ significa exatamente dizer que existe um elemento $\psi\in(\mathbb{Z}/p\mathbb{Z})^\times$ com ordem $\operatorname{ord}(\psi)=2N$ tal que $\psi^{2N}\equiv 1\pmod p$ e nenhuma potência menor vale $1$.

Como em qualquer grupo finito a ordem de um elemento divide a ordem do grupo, necessariamente deve valer
\[
2N \mid |(\mathbb{Z}/p\mathbb{Z})^\times| = p-1.
\]
Reciprocamente, como $(\mathbb{Z}/p\mathbb{Z})^\times$ é cíclico, para todo divisor $d$ de $p-1$ existe um elemento de ordem exatamente $d$; em particular, se $2N\mid(p-1)$ então existe $\psi$ com $\operatorname{ord}(\psi)=2N$.

Concluímos, portanto, que existe uma raiz primitiva $2N$-ésima da unidade em $\mathbb{Z}/p\mathbb{Z}$ se, e somente se, $2N\mid(p-1)$ (equivalentemente $p\equiv 1 \pmod{2N}$).

\hrulefill

Dados esses critérios de existência, trataremos da estrutura algébrica da transformada. A NTT Negacíclica é definida no anel quociente 
\[
R = \frac{\mathbb{Z}_p[x]}{(x^N+1)}
\]

ou seja, há uma composição de operações modulares. (A imagem \ref{fig:torus} ilustra a sequência das duas operações visualmente.)

Ela pode ser vista como a avaliação do polinômio nas raízes da equação $x^N + 1 = 0$, que correspondem às potências ímpares de $\psi$. A transformada é definida por:

\[
X[k] = \sum_{n=0}^{N-1} x[n] \psi^{(2k+1)n} \pmod{p}
\]

Na forma matricial:

\[
\mathbf{X} = \mathbf{W}_N \, \mathbf{x}, \qquad \mathbf{x} = 
\begin{bmatrix} 
x_0 \\ x_1 \\ \vdots \\ x_{N-1} 
\end{bmatrix}
\]

Neste caso, a matriz de transformação $\mathbf{W}_N$ difere da versão cíclica padrão, pois seus coeficientes seguem a estrutura das raízes negacíclicas:

\[  
\mathbf{W}_N = 
\begin{bmatrix}
1 & \psi^1 & \psi^2 & \cdots & \psi^{(N-1)} \\
1 & \psi^{3} & \psi^{6} & \cdots & \psi^{3(N-1)} \\
1 & \psi^{5} & \psi^{10} & \cdots & \psi^{5(N-1)} \\
\vdots & \vdots & \vdots & \ddots & \vdots \\
1 & \psi^{(2k+1)} & \psi^{(2k+1)2} & \cdots & \psi^{(2k+1)(N-1)} \\
\vdots & \vdots & \vdots & \ddots & \vdots \\
1 & \psi^{(2N-1)} & \psi^{(2N-1)2} & \cdots & \psi^{(2N-1)(N-1)}
\end{bmatrix}
\]

De forma geral, o termo na linha $k$ e coluna $n$ da matriz é dado por:
\[
(\mathbf{W}_N)_{k,n} = \psi^{(2k+1)n} \pmod{p}, \qquad 0 \le k,n \le N-1
\]


\begin{figure}[H]
    \centering
\includegraphics{images/torus.pdf}
\caption{representação das transformações}
\label{fig:torus}
\end{figure}


\subsection{O Isomorfismo via Teorema Chinês dos Restos (CRT)}

A fundamentação algébrica da NTT reside na estrutura do anel quociente
$\mathbb{Z}_p[x]/(x^N+1)$.
Já que $x^N+1$ fatora-se em $N$ fatores lineares distintos e coprimos,
\[
x^N+1=\prod_{k=0}^{N-1}\bigl(x-\psi^{2k+1}\bigr),
\]

\medskip
\noindent
\textbf{Homomorfismo de avaliação.}

Para cada $\alpha\in\mathbb{Z}_p$, considere o mapa de avaliação
\[
\mathrm{ev}_\alpha:\mathbb{Z}_p[x]\to \mathbb{Z}_p,
\qquad
\mathrm{ev}_\alpha(a(x))=a(\alpha).
\]
Esse mapa é um \emph{homomorfismo de anéis} (preserva soma e produto):
\[
\mathrm{ev}_\alpha(a+b)=\mathrm{ev}_\alpha(a)+\mathrm{ev}_\alpha(b),
\qquad
\mathrm{ev}_\alpha(ab)=\mathrm{ev}_\alpha(a)\,\mathrm{ev}_\alpha(b).
\]
Além disso, seu núcleo é exatamente o ideal gerado por $(x-\alpha)$:
\[
\ker(\mathrm{ev}_\alpha)=(x-\alpha).
\]
De fato, pela Divisão Euclidiana, qualquer $a(x)$ pode ser escrito como
\[
a(x)=q(x)(x-\alpha)+r
\]
com $r\in\mathbb{Z}_p$, e avaliando em $x=\alpha$ obtemos $a(\alpha)=r$, uma vez que $(x-\alpha)$ zera.
Logo $a(\alpha)=0$ se e somente se $(x-\alpha)\mid a(x)$.
Pelo \textbf{Primeiro Teorema do Isomorfismo} para anéis,
\[
\frac{\mathbb{Z}_p[x]}{(x-\alpha)}\;\cong\;\mathrm{Im}(\mathrm{ev}_\alpha)=\mathbb{Z}_p,
\qquad
[a(x)]\longmapsto a(\alpha).
\]

\medskip
\noindent
\textbf{O isomorfismo global do CRT como produto de avaliações.}

Como os fatores $(x-\psi^{2k+1})$ são coprimos dois a dois, o CRT fornece um
isomorfismo de anéis
\[
\Phi:\frac{\mathbb{Z}_p[x]}{(x^N+1)}
\;\xrightarrow{\cong}\;
\prod_{k=0}^{N-1}\frac{\mathbb{Z}_p[x]}{\bigl(x-\psi^{2k+1}\bigr)}
\;\cong\;
\underbrace{\mathbb{Z}_p\times\cdots\times\mathbb{Z}_p}_{N\ \text{vezes}}.
\]
E, crucialmente, $\Phi$ é construído a partir dos homomorfismos de avaliação:
a $k$-ésima componente é exatamente o homomorfismo
$\mathrm{ev}_{\psi^{2k+1}}$ (passando ao quociente).
Assim, para $[a(x)]\in\mathbb{Z}_p[x]/(x^N+1)$,
\[
\Phi([a(x)])=
\bigl(
a(\psi^{1}),\, a(\psi^{3}),\,\dots,\, a(\psi^{2N-1})
\bigr)\in\mathbb{Z}_p^N.
\]

\medskip
\noindent
\textbf{Multiplicação vira produto ponto a ponto.}
Como cada $\mathrm{ev}_{\psi^{2k+1}}$ é homomorfismo e $\Phi$ é isomorfismo de anéis,
\[
\Phi([a(x)]\,[b(x)])=\Phi([a(x)])\cdot \Phi([b(x)])
\]
onde o produto do lado direito é componente a componente. Em particular, se
$C=\Phi([a(x)b(x)])$, então
\[
C_k=(ab)(\psi^{2k+1})=a(\psi^{2k+1})\,b(\psi^{2k+1}).
\]
Isso explica formalmente por que a multiplicação no quociente
$\mathbb{Z}_p[x]/(x^N+1)$ (convolução negacíclica dos coeficientes)
se traduz em um produto de Hadamard no domínio transformado.


\begin{figure}[H]
    \centering
    % --- Primeira Imagem (Esquerda) ---
    \begin{minipage}{0.48\textwidth}
        \centering
        \includegraphics[width=\linewidth]{images/CRT_NTT4.pdf}
        \caption{CRT da NTT Negacíclica (Exemplo N=4)}
        \label{fig:NTT4}
    \end{minipage}\hfill % O \hfill cria o espaço entre as figuras
    % --- Segunda Imagem (Direita) ---
    \begin{minipage}{0.48\textwidth}
        \centering
        \includegraphics[width=\linewidth]{images/CRT_NTT.pdf}
        \caption{CRT da NTT Negacíclica (Geral)}
        \label{fig:NTT_Geral}
    \end{minipage}
\end{figure}

\subsection{Transformada Numérica Inversa (INTT)}

Como o Teorema Chinês dos Restos garante que a aplicação da NTT é um isomorfismo bijetor entre o anel de polinômios $\mathbb{Z}_p[x]/(x^N+1)$ e o domínio da frequência, existe uma transformação inversa única capaz de recuperar os coeficientes originais.

Denotamos a inversa multiplicativa de $N$ no corpo $\mathbb{Z}_p$ por $N^{-1}$, tal que $N \cdot N^{-1} \equiv 1 \pmod{p}$. A NTT Negacíclica Inversa (INTT) é definida formalmente por:

\[
x[n] = N^{-1} \sum_{k=0}^{N-1} X[k] \psi^{-(2k+1)n} \pmod{p}
\]

Note que o termo $\psi^{-(2k+1)n}$ refere-se à potência do inverso multiplicativo da raiz.
\subsection*{Representação Matricial}
Na forma matricial, a operação de inversão corresponde à resolução do sistema linear $\mathbf{X} = \mathbf{W}_N \mathbf{x}$. A solução é dada por:

\[
\mathbf{x} = \mathbf{W}_N^{-1} \mathbf{X}
\]

Onde a matriz inversa $\mathbf{W}_N^{-1}$ é definida como o inverso modular das potências ímpares de $\psi$, escalada pelo fator $N^{-1}$:

\[
\mathbf{W}_N^{-1} = N^{-1}
\begin{bmatrix}
 1 & 1 & 1 & \cdots & 1 \\
 \psi^{-1} & \psi^{-3} & \psi^{-5} & \cdots & \psi^{-(2N-1)} \\
 \psi^{-2} & \psi^{-6} & \psi^{-10} & \cdots & \psi^{-2(2N-1)} \\
 \vdots & \vdots & \vdots & \ddots & \vdots \\
 \psi^{-n} & \psi^{-3n} & \psi^{-5n} & \cdots & \psi^{-n(2k+1)} \\
 \vdots & \vdots & \vdots & \ddots & \vdots \\
 \psi^{-(N-1)} & \psi^{-3(N-1)} & \psi^{-5(N-1)} & \cdots & \psi^{-(N-1)(2N-1)}
\end{bmatrix}
\]

De modo geral, o termo na linha $n$ e coluna $k$ da matriz inversa é:
\[
(\mathbf{W}_N^{-1})_{n,k} = N^{-1} \psi^{-(2k+1)n} \pmod{p}
\]


\subsection{A NTT como uma FFT: Decomposição Radix-2   }

A eficiência da NTT negacíclica baseia-se na estratégia \emph{dividir para conquistar}, análoga à FFT clássica (algoritmo de Cooley-Tukey). Em vez de avaliar diretamente um polinômio $A(x)$ em $N$ pontos, decompomos o problema em dois subproblemas de tamanho $N/2$ através da separação dos coeficientes em índices pares e ímpares.

\subsubsection*{1. Definições e Propriedades}
Seja $N=2^m$ e $p$ um primo tal que exista uma raiz primitiva $\psi \in \mathbb{Z}_p^*$ de ordem $2N$. As propriedades fundamentais para a aritmética negacíclica são:
\[
\psi^{2N} \equiv 1 \pmod p \quad \text{e} \quad \psi^{N} \equiv -1 \pmod p.
\]
A NTT negacíclica consiste na avaliação do polinômio $A(x)$ nas $N$ raízes ímpares da unidade:
\[
x_k = \psi^{2k+1}, \quad \text{para } k=0, 1, \dots, N-1.
\]

\subsubsection*{2. Decomposição do Polinômio (Decimation-in-Time)}
Podemos reescrever $A(x)$ separando os termos com potências pares e ímpares de $x$:
\begin{align*}
A(x) &= \sum_{i=0}^{N-1} a[i]x^i \\
     &= \sum_{i=0}^{N/2-1} a[2i]x^{2i} + \sum_{i=0}^{N/2-1} a[2i+1]x^{2i+1} \\
     &= \sum_{i=0}^{N/2-1} a[2i](x^2)^i + x \cdot \sum_{i=0}^{N/2-1} a[2i+1](x^2)^i.
\end{align*}
Definindo os polinômios auxiliares $A_{\text{par}}(y)$ e $A_{\text{impar}}(y)$ como as partes pares e ímpares, respectivamente, obtemos a relação recursiva:
\[
A(x) = A_{\text{par}}(x^2) + x \cdot A_{\text{impar}}(x^2).
\]

\subsubsection*{3. A Recursão e o Fator $\psi^2$}
Ao avaliarmos essa expressão nos pontos $x_k = \psi^{2k+1}$, observamos o comportamento do termo quadrático:
\[
x_k^2 = (\psi^{2k+1})^2 = \psi^{4k+2} = (\psi^2)^{2k+1}.
\]
Isso revela que, para os subproblemas de tamanho $N/2$, a base da transformação torna-se $\psi^2$. Como a ordem de $\psi$ é $2N$, a ordem de $\psi^2$ é $N$, satisfazendo o requisito de raiz primitiva para o subgrupo de tamanho $N/2$.

Logo, os valores retornados pelas chamadas recursivas são:
\[
Y_{\text{par}}[k] = A_{\text{par}}(x_k^2) \quad \text{e} \quad Y_{\text{impar}}[k] = A_{\text{impar}}(x_k^2).
\]

\subsubsection*{4. A Borboleta (Butterfly) Negacíclica}
Para recombinar os resultados e obter $Y[k] = A(x_k)$, exploramos a simetria das raízes.
Lembre que, após a decomposição $A(x)=A_{\text{par}}(x^2) + x\,A_{\text{impar}}(x^2)$, avaliamos em
\[
x_k = \psi^{2k+1}, \qquad k=0,1,\dots,N-1,
\]
de modo que
\[
Y_{\text{par}}[k] = A_{\text{par}}(x_k^2), 
\qquad
Y_{\text{impar}}[k] = A_{\text{impar}}(x_k^2),
\qquad (0\le k < N/2).
\]

Para a primeira metade ($0 \le k < N/2$), a recombinação é direta.
Para a segunda metade, usamos a simetria negacíclica das raízes ímpares:
\[
x_{k+N/2} 
= \psi^{2(k+N/2)+1} 
= \psi^{2k+1}\cdot \psi^N
= -\psi^{2k+1}
= -x_k.
\]

Assim, o termo $x\cdot A_{\text{impar}}(x^2)$ apenas troca de sinal na segunda metade. As equações da
borboleta tornam-se:

\[
\boxed{
\begin{aligned}
Y[k] &= Y_{\text{par}}[k] + x_k \cdot Y_{\text{impar}}[k] \pmod p \\
Y[k+N/2] &= Y_{\text{par}}[k] - x_k \cdot Y_{\text{impar}}[k] \pmod p
\end{aligned}
}
\]

\textbf{Nota de Implementação:}
Na prática, os valores $x_k=\psi^{2k+1}$ não são recalculados por exponenciação.
Percorremos as raízes ímpares multiplicando por $\psi^2$ a cada passo:
inicialize $x\gets \psi$ e atualize $x \gets x\cdot \psi^2$, gerando
$(\psi^1,\psi^3,\psi^5,\dots)$.

\begin{figure}[H]
     \centering
     \begin{subfigure}[b]{0.48\textwidth}
         \centering
         \includegraphics[width=\textwidth]{images/butterflyNTT.pdf}
         \caption{Butterfly NTT}
         \label{fig:bntt}
     \end{subfigure}
     \hfill
     \begin{subfigure}[b]{0.48\textwidth}
         \centering
         \includegraphics[width=\textwidth]{images/butterflyNTT_4.pdf}
         \caption{Butterfly NTT 4}
         \label{fig:b4ntt}
     \end{subfigure}
     \label{fig:comparison_ntt}
\end{figure}

\subsection{A INTT via Gentleman--Sande (GS): Reconstrução }

Enquanto a transformação direta (NTT) utiliza \emph{dividir para conquistar} separando coeficientes pares e ímpares,
a transformação inversa (INTT) percorre o caminho oposto: combina resultados de subproblemas menores para recuperar
o vetor original. O esquema de Gentleman--Sande pode ser visto como a versão “transposta” (em ordem de operações)
da decomposição de Cooley--Tukey.

\subsubsection*{1. Paridade e o sistema linear local}
Na NTT negacíclica por partição par/ímpar, escrevemos
\[
A(x)=A_e(x^2)+x\,A_o(x^2),
\qquad
A_e(t)=\sum_{i=0}^{N/2-1} a[2i]t^i,\quad
A_o(t)=\sum_{i=0}^{N/2-1} a[2i+1]t^i.
\]
A avaliação ocorre nos pontos ímpares $x_k=\psi^{2k+1}$, $k=0,\dots,N-1$.
Denotando por $Y_e[k]=A_e(x_k^2)$ e $Y_o[k]=A_o(x_k^2)$ (retornados pela recursão), a combinação (borboleta direta) é:
\[
\begin{cases}
Y[k] = Y_e[k] + x_k\,Y_o[k],\\
Y[k+N/2] = Y_e[k] - x_k\,Y_o[k],
\end{cases}
\qquad 0\le k < N/2.
\]
Na etapa inversa, conhecemos $Y[k]$ e $Y[k+N/2]$ e queremos recuperar $Y_e[k]$ e $Y_o[k]$.

\subsubsection*{2. A lógica da recuperação (inversão do 2$\times$2)}
Somando e subtraindo as equações acima, obtemos:
\[
Y[k]+Y[k+N/2] = 2Y_e[k],
\qquad
Y[k]-Y[k+N/2] = 2x_k\,Y_o[k].
\]
Como $p$ é primo ímpar, $2$ é invertível em $\mathbb{Z}_p$ e podemos escrever:
\[
\boxed{
\begin{aligned}
Y_e[k] &= 2^{-1}\bigl(Y[k]+Y[k+N/2]\bigr)\pmod p,\\[2mm]
Y_o[k] &= 2^{-1}\bigl(Y[k]-Y[k+N/2]\bigr)\,x_k^{-1}\pmod p.
\end{aligned}}
\]
Assim, a borboleta inversa recupera os dois vetores de entrada do nível, que então alimentam as recursões
de tamanho $N/2$.

\subsubsection*{3. Por que a recursão usa $\psi^2$ novamente}
Na ida, ao avaliar em $x_k=\psi^{2k+1}$, temos
\[
x_k^2=(\psi^{2k+1})^2=\psi^{4k+2}=(\psi^2)^{2k+1},
\]
ou seja, os subproblemas são novamente NTTs negacíclicas em $N/2$ pontos ímpares, mas agora com parâmetro $\psi^2$.
Na volta, o mesmo vale: após aplicar a borboleta inversa no nível corrente, chamamos recursivamente a INTT
nos blocos correspondentes usando $\psi^{-2}$.

\subsubsection*{4. Ordem das operações: CT vs GS}
A diferença estrutural entre as duas direções pode ser resumida assim:
\begin{enumerate}
    \item Na \textbf{Ida (CT / DIT)}, primeiro resolvemos recursivamente e \emph{depois} combinamos com o fator $x_k$.
    \item Na \textbf{Volta (GS)}, primeiro desfazemos a combinação local (somar/subtrair e multiplicar por $x_k^{-1}$)
    e \emph{depois} prosseguimos recursivamente.
\end{enumerate}

\textit{Nota de implementação:} o fator $2^{-1}$ pode ser acumulado ao longo das camadas e aplicado como um único
fator global $N^{-1}$ ao final, reduzindo o número de multiplicações escalares.
Além disso, para evitar criar uma variável dedicada ao \emph{twiddle factor}, usa-se diretamente $x_k^{-1}$
(ou o valor já pré-computado em uma tabela de potências) no ponto em que a multiplicação é necessária.



\begin{figure}[H]
    \centering
    \includegraphics{images/INTT_butterfly.pdf}
    \caption{Butterfly INTT}
    \label{fig:bINNT}
\end{figure}
A implementa da NTT negacíclica está  no pseudocódigo a seguir
\begin{figure}[H]
    \centering
    % --- ALGORITMO À ESQUERDA (FORWARD NTT) ---
    \begin{minipage}{0.49\textwidth}
        \hrule
        \vskip 2pt
        \captionof{algorithm}{NTT Negacíclica (CT)}
        \hrule
        \begin{algorithmic}[1]
            \footnotesize
            \Require $a \in \mathbb{Z}_p^N$ (Coeficientes de entrada)
            \Require $N = 2^m $e Módulo $p$ 
            \Require $\psi$: Raiz primitiva $2N$-ésima da unidade
            
            \State $len \gets N/2$
            \For{$l \gets m-1$ \textbf{down to} $0$}
                \State $num\_gr \gets 2^{m-1-l}$
                \For{$i \gets 0$ \textbf{to} $num\_gr - 1$}
                    \State $\zeta \gets \psi^{\text{brv}_m(num\_gr + i)}$
                    \For{$j \gets i \cdot 2^{l+1}$ \textbf{to} $... + 2^l - 1$}
                        \State $t_0 \gets a[j]$
                        \State $t_1 \gets \zeta \cdot a[j + 2^l] \pmod p$
                        \State $a[j] \gets t_0 + t_1 \pmod p$
                        \State $a[j + 2^l] \gets t_0 - t_1 \pmod p$
                        \State
                    \EndFor
                    \State
                \EndFor
                \State
            \EndFor
            \State \Return $a$ (\emph{bit-reversed})
        \end{algorithmic}
        \hrule
    \end{minipage}
    \hfill
    % --- ALGORITMO À DIREITA (INVERSE NTT) ---
    \begin{minipage}{0.49\textwidth}
        \hrule
        \vskip 2pt
        \captionof{algorithm}{INTT (GS)}
        \hrule
        \begin{algorithmic}[1]
            \footnotesize
            \Require $\hat{a}$ (Vetor em ordem \emph{bit-reversed})
            \Require $N = 2^m$ e Módulo $p$
            \Require $\psi$: Raiz primitiva $2N$-ésima da unidade
            
            \For{$l \gets 0$ \textbf{to} $m-1$}
                \State $num\_gr \gets 2^{m-1-l}$
                \For{$i \gets 0$ \textbf{to} $num\_gr - 1$}
                    \State $\zeta \gets \psi^{-\text{brv}_m(num\_gr + i)}$
                    \For{$j \gets i \cdot 2^{l+1}$ \textbf{to} $... + 2^l - 1$}
                        \State $t_0 \gets a[j]$
                        \State $t_1 \gets a[j + 2^l]$
                        \State $a[j] \gets t_0 + t_1 \pmod p$
                        \State $a[j + 2^l] \gets (t_0 - t_1)\zeta \pmod p$
                    \EndFor
                \EndFor
            \EndFor
            \State $invN \gets N^{-1} \pmod p$
            \For{$i \gets 0$ \textbf{to} $N-1$}
                \State $a[i] \gets a[i] \cdot invN$
            \EndFor
            \State \Return $a$ (Ordem normal)
        \end{algorithmic}
        \hrule
    \end{minipage}
\end{figure}

\section{Testes comparativos}
A comparação computacional dessas ferramentas pode ser vista na tabela a seguir, onde calculou-se o maior número de Fibonacci obtido em menos de um segundo. Todos os testes foram executados em single-core.

Nos algoritmos da FFT e da NTT, o n-ésimo número de Fibonacci foi calculado pela seguinte relação:
\[\begin{bmatrix}
    F_{n+1}\\
    F_{n}
\end{bmatrix}
=
\begin{bmatrix}
    1&1\\
    1&0
\end{bmatrix}^n
\cdot
\begin{bmatrix}
    F_1\\
    F_0
\end{bmatrix}
\]
sendo a multiplicação que ocorre tanto na exponenciação quanto no produto interno foram otimizadas com o devido método.

Além disso, como o foco foi apenas em comparar a velocidade dos métodos, utilizou-se de scripts para a geração de macros com os valores das raízes primitivas, raízes da unidade, entre outras constantes; para mais informações, veja o repositório.


(GMP é uma biblioteca otimizada para operações com inteiros extremamente grandes)

\textbf{Setup:}

\textbf{CPU:} 13th Gen Intel(R) Core(TM) i7-1365U (12) @ 5.20 GHz

\begin{table}[ht]
    \centering
    \caption{Comparação de Desempenho dos Algoritmos}
    \label{tab:comparacao_algoritmos}
    \begin{tabular}{lr} 
        \toprule
        \textbf{Algoritmo} & \textbf{Quantidade nº  calculados (\%)} \\
        \midrule
        Algoritmo Naive   & 44  \\ 
        Algoritmo Iterado & 566.053  \\ 
        Algoritmo FFT     & 3.145.816  \\ 
        Algoritmo NTT     & 24.178.839 \\
        Algoritmo GMP     & 238.961.323  \\
        \bottomrule
    \end{tabular}
    
\end{table}
fonte: \url{https://github.com/SheafificationOfG/Fibsonisheaf}

\section{NTT incompleta e a restrição de parâmetros no caso negacíclico}

Na multiplicação negacíclica de polinômios no anel
\[
\mathbb{Z}_p[x]/(x^N + 1),
\]
com $N$ potência de dois, a utilização de uma \textbf{NTT completa} exige a existência
de uma raiz primitiva de ordem $2N$ módulo $p$. Quando $p$ é primo, essa condição é
equivalente a
\[
p \equiv 1 \pmod{2N}.
\]
Em aplicações criptográficas baseadas em reticulados, em especial em esquemas de Fully Homomorphic Encryption,
o valor de $N$ é tipicamente muito grande. Como consequência, a condição acima impõe
fortes restrições sobre a escolha do primo $p$, frequentemente forçando o uso de
módulos grandes ou pouco flexíveis, o que impacta tanto a eficiência quanto o ajuste
fino de parâmetros de segurança.

Uma forma natural de relaxar essa restrição é empregar a chamada \textbf{NTT
$\ell$-incompleta}. A ideia consiste em interromper o algoritmo da NTT antes de
executar todos os $\log_2 N$ estágios do esquema radix-2. Mais precisamente, ao parar
após $\log_2 N - \ell$ estágios, a existência da transformada passa a requerer apenas
uma raiz da unidade de ordem $2N/2^{\ell}$ em $\mathbb{Z}_p$, o que resulta na condição
mais fraca
\[
p \equiv 1 \pmod{2N/2^{\ell}}.
\]
\begin{figure}
    \centering
    \includegraphics{images/ntt_incompleta.pdf}
    \caption{árvore de fracionamento da NTT }
    \label{fig:incompleta}
\end{figure}
Dessa forma, o conjunto de primos admissíveis torna-se significativamente maior,
permitindo escolhas de parâmetros mais flexíveis.

Do ponto de vista algébrico, a NTT $\ell$-incompleta não avalia o polinômio em todas as
$N$ raízes da unidade, mas o mapeia para um vetor de $N/2^{\ell}$ polinômios de menor
grau, cada um pertencente a um quociente do tipo
\[
\mathbb{Z}_p[x]/(x^{2^{\ell}} - \psi_i),
\]
onde $\psi_i$ são potências apropriadas da raiz disponível. A multiplicação passa então
a ser realizada componente a componente nesses anéis menores, seguida de uma
transformada inversa incompleta.

Para demonstrar o impacto prático dessa flexibilidade algébrica, analisa-se o cenário de
\textit{bootstrapping} amortizado em FHE. Como a complexidade das operações aritméticas
depende diretamente do tamanho do módulo $p$, a capacidade de selecionar primos
significativamente menores — viabilizada pela incompletude da NTT — traduz-se em
ganhos de desempenho mensuráveis. A Tabela \ref{tab:ntt_bootstrapping_comparison}
quantifica essa melhoria, contrastando os custos computacionais e a escolha ótima de
parâmetros para diferentes níveis de segurança em comparação com a restrição rígida
imposta pela NTT completa.

\begin{table}[ht]
    \centering
    \caption{Comparação de Desempenho: Bootstrapping com NTT Incompleta vs. Completa}
    \label{tab:ntt_bootstrapping_comparison}
    \begin{tabular}{ccccc}
        \toprule
        \textbf{Nível de} & \textbf{Grau Ótimo de} & \textbf{Módulo Ótimo} & \textbf{Módulo NTT} & \textbf{Ganho de} \\
        \textbf{Segurança ($\lambda$)} & \textbf{Incompletude ($\ell$)} & \textbf{($p$, Incompleta)} & \textbf{Completa ($\ell=10$)} & \textbf{Desempenho} \\
        \midrule
        128 & 6 & 3.329  & 12.289 & 33\% \\
        192 & 8 & 7.681  & 12.289 & 29\% \\
        256 & 8 & 12.289 & 12.289 & 6\%  \\
        512 & 7 & 18.433 & 40.961 & 42\% \\
        \bottomrule
    \end{tabular}
    
    \vspace{0.2cm}
    \small
    \raggedright
    \textbf{Nota:} Para $\lambda=256$, o módulo ótimo permanece o mesmo ($q=12.289$) em ambos os cenários, resultando em um ganho de desempenho menor comparado aos outros níveis onde a redução do módulo é significativa.
\end{table}

\begin{figure}[ht]
    \centering
    % --- Primeira Figura (n=256) ---
    \begin{subfigure}[b]{0.32\textwidth}
        \centering
        \includegraphics[width=\linewidth]{images/n=256}
        \caption{Resultados para $n=256$}
        \label{fig:n256}
    \end{subfigure}
    \hfill % Espaçamento flexível entre as figuras
    % --- Segunda Figura (n=512) ---
    \begin{subfigure}[b]{0.32\textwidth}
        \centering
        \includegraphics[width=\linewidth]{images/n=512}
        \caption{Resultados para $n=512$}
        \label{fig:n512}
    \end{subfigure}
    \hfill % Espaçamento flexível entre as figuras
    % --- Terceira Figura (n=1024) ---
    \begin{subfigure}[b]{0.32\textwidth}
        \centering
        \includegraphics[width=\linewidth]{images/n=1024}
        \caption{Resultados para $n=1024$}
        \label{fig:n1024}
    \end{subfigure}
    
    \caption{Imagens retiradas do artigo \cite{11129329} }
    \label{fig:comparacao_n}
\end{figure}
Os resultados experimentais apresentados na Figura \ref{fig:comparacao_n} mostram o ganho percentual e eles sugerem que a adoção de valores maiores de $n$ em esquemas FHE (por exemplo, $n \ge 1024$) pode limitar a melhoria de desempenho obtida pela incompletude da NTT. 
Observa-se que o número de instâncias onde as configurações de NTT incompleta apresentam melhor desempenho diminui à medida que $n$ aumenta. 
Frequentemente, as restrições de segurança para esses casos exigem a utilização de módulos primos maiores (como $q=12289$), cenário no qual a NTT completa acaba por ser a escolha ótima, anulando os ganhos que seriam obtidos através do uso de módulos menores permitidos pela técnica incompleta.

\section{Agradecimentos}
Gostaria de agradecer ao professor doutor \emph{Thales Paiva} pela chance de apresentar o seminário e pelos ensinamentos,
gostaria de agradecer também o professor doutor \emph{João Fernando da Cunha Nariyoshi} por esclarecer algumas dúvidas sobre a parte da matemática pura.
Foi realmente divertido pesquisar, escrever, desenhar e animar para esse seminário.

\section{Apêndice}
\subsection{DFT como transformação linear}
\label{sec:linear}

Seja $\zeta = e^{-2\pi i/N}$ uma raiz primitiva $N$-ésima da unidade. Definimos a Transformada Discreta de Fourier (DFT) como a operação que mapeia um sinal discreto $x[n]$ de comprimento $N$ em uma sequência $X[k]$ no domínio da frequência, dada por:
\[
    X[k] = \sum_{n=0}^{N-1} x[n]\;\zeta^{kn}, \quad k=0,1,\dots,N-1.
\]

\begin{teorema}
    A Transformada Discreta de Fourier é uma transformação linear. Ou Seja, para quaisquer sinais $x[n], y[n]$ e escalares $a, b \in \mathbb{C}$, vale:
    \[
        \mathrm{DFT}(a x + b y) = a\,\mathrm{DFT}(x) + b\,\mathrm{DFT}(y).
    \]
\end{teorema}

\begin{proof}
    Sejam $x[n]$ e $y[n]$ dois sinais e $a, b \in \mathbb{C}$ constantes. Definimos a combinação linear $z[n] = a\,x[n] + b\,y[n]$. Aplicando a definição da DFT ao sinal $z[n]$, obtemos $Z[k]$:
    \[
        Z[k] = \sum_{n=0}^{N-1} z[n]\;\zeta^{kn} = \sum_{n=0}^{N-1} (a\,x[n] + b\,y[n])\;\zeta^{kn}.
    \]
    
    Pela propriedade distributiva do somatório e pela linearidade da soma em $\mathbb{C}$:
    \[
        Z[k] = \sum_{n=0}^{N-1} a\,x[n]\zeta^{kn} + \sum_{n=0}^{N-1} b\,y[n]\zeta^{kn}.
    \]
    
    Como os escalares $a$ e $b$ não dependem do índice da soma $n$, podemos fatorá-los para fora dos somatórios:
    \[
        Z[k] = a \sum_{n=0}^{N-1} x[n]\zeta^{kn} + b \sum_{n=0}^{N-1} y[n]\zeta^{kn}.
    \]
    
    Reconhecendo as expressões dentro dos somatórios como as definições de $X[k]$ e $Y[k]$, respectivamente, concluímos que:
    \[
        Z[k] = a\,X[k] + b\,Y[k].
    \]
    
    Portanto, a relação de linearidade é satisfeita para cada componente $k$, o que prova que a DFT é uma transformação linear.
\end{proof}

\subsection{Exemplo de convolução negacíclica}
\label{sec:convolução}
Sejam dois sinais $x$ e $y$ de comprimento $N=3$:
\[
x = \begin{bmatrix} 1 \\ 2 \\ 0 \end{bmatrix}, \quad y = \begin{bmatrix} 1 \\ 0 \\ 1 \end{bmatrix}
\]

Neste contexto, a frequência fundamental não é a raiz da unidade, mas sim a raiz de $x^N + 1 = 0$. Seja $\psi$ tal que $\psi^N = -1$. Para $N=3$, temos $\psi^3 = -1$.

\subsubsection*{Método 1: Convolução no Tempo (Matriz Negacirculante)}

A convolução negacíclica $z = x *_{neg} y$ equivale à multiplicação de uma matriz negacirculante $H_x$ pelo vetor $y$. Note que os elementos que ``dão a volta'' (acima da diagonal principal) têm sinal invertido:

\[
H_x = \begin{bmatrix} 
x[0] & -x[2] & -x[1] \\ 
x[1] & x[0] & -x[2] \\ 
x[2] & x[1] & x[0] 
\end{bmatrix} 
= \begin{bmatrix} 
1 & 0 & -2 \\ 
2 & 1 & 0 \\ 
0 & 2 & 1 
\end{bmatrix}
\]

Calculando $z = H_x y$:
\[
z = \begin{bmatrix} 1 & 0 & -2 \\ 2 & 1 & 0 \\ 0 & 2 & 1 \end{bmatrix} 
\begin{bmatrix} 1 \\ 0 \\ 1 \end{bmatrix} 
= \begin{bmatrix} 
1(1) + 0 - 2(1) \\ 
2(1) + 0 + 0 \\ 
0 + 0 + 1(1) 
\end{bmatrix} 
= \begin{bmatrix} -1 \\ 2 \\ 1 \end{bmatrix}
\]

\subsubsection*{Método 2: Diagonalização (Domínio da Frequência)}

A matriz negacirculante $H_x$ é diagonalizada por uma variante da DFT que utiliza potências ímpares de $\psi$ (raízes de $x^N+1$). Matematicamente:
$$H_x = \mathcal{F}^{-1} \Lambda_x \mathcal{F}$$
onde $\Lambda_x$ contém os autovalores de $H_x$, que correspondem à avaliação do polinômio $x(z)$ nas raízes de $z^N = -1$.

Para verificar os autovalores (o espectro de $x$), calculamos o polinômio característico de $H_x$:
\[
\det(H_x - \lambda I) = 0 \implies 
\det \begin{bmatrix} 
1-\lambda & 0 & -2 \\ 
2 & 1-\lambda & 0 \\ 
0 & 2 & 1-\lambda 
\end{bmatrix} = 0
\]
Expandindo o determinante (regra de Sarrus):
\[
(1-\lambda)^3 - (2)(2)(2) = 0 \implies (1-\lambda)^3 = 8
\]
As raízes para $(1-\lambda)$ são as raízes cúbicas de $8$. Sabemos que $8 = 8 \cdot 1$, mas no contexto complexo as raízes são $2$, $2\zeta_3$ e $2\zeta_3^2$. Logo:
\begin{align*}
1 - \lambda_0 = 2 &\implies \lambda_0 = -1 \\
1 - \lambda_1 = 2\zeta_3 &\implies \lambda_1 = 1 - 2\zeta_3 \\
1 - \lambda_2 = 2\zeta_3^2 &\implies \lambda_2 = 1 - 2\zeta_3^2
\end{align*}
Estes valores $\lambda_k$ correspondem à Transformada Negacíclica (NTT sobre $x^N+1$) do vetor $x$, permitindo calcular a convolução via produto ponto a ponto.

\subsubsection*{Verificação do Autovetor}
Verificamos agora se o autovetor $v_1$ da base Negacíclica (coluna de $\mathcal{F}^{-1}$ associada à raiz $\psi$), dado por $v_1 = [1, \psi^{-1}, \psi^{-2}]^T = [1, -\psi^2, -\psi]^T$, satisfaz $H_x v_1 = \lambda_1 v_1$.
\\
Lado esquerdo ($H_x v_1$):
\[
\begin{bmatrix} 1 & 0 & -2 \\ 2 & 1 & 0 \\ 0 & 2 & 1 \end{bmatrix} 
\begin{bmatrix} 1 \\ -\psi^2 \\ -\psi \end{bmatrix} 
= 
\begin{bmatrix} 
1 + 2\psi \\ 
2 - \psi^2 \\ 
-2\psi^2 - \psi 
\end{bmatrix}
\]
Lado direito ($\lambda_1 v_1$), onde o autovalor é $\lambda_1 = x(\psi) = 1+2\psi$:
\[
(1+2\psi) 
\begin{bmatrix} 1 \\ -\psi^2 \\ -\psi \end{bmatrix} 
=
\begin{bmatrix} 
1 + 2\psi \\ 
-\psi^2 - 2\psi^3 \\ 
-\psi - 2\psi^2 
\end{bmatrix}
\]
Utilizando a propriedade $\psi^3 = -1$, simplificamos o termo do meio: $-\psi^2 - 2(-1) = 2 - \psi^2$. A igualdade é satisfeita, confirmando que as raízes de $x^N+1$ geram a base natural de $H_x$.

\subsubsection*{Representação Matricial da Diagonalização}

Primeiro, calculamos os vetores transformados $X = \text{NTT}(x)$ e $Y = \text{NTT}(y)$ avaliando os polinômios nas raízes de $z^3=-1$:
\[
X = \begin{bmatrix} x(-1) \\ x(\psi) \\ x(\psi^2) \end{bmatrix} 
= \begin{bmatrix} -1 \\ 1+2\psi \\ 1+2\psi^2 \end{bmatrix}, \quad
Y = \begin{bmatrix} y(-1) \\ y(\psi) \\ y(\psi^2) \end{bmatrix} 
= \begin{bmatrix} 2 \\ 1+\psi^2 \\ 1+\psi^4 \end{bmatrix}
\]
\small{\textit{*Nota: $y(\psi) = 1 + 0\psi + 1\psi^2$. Para o terceiro termo, usamos $\psi^4 = -\psi$, logo $1-\psi$.}}

Agora, construímos a matriz diagonal $\Lambda_x = \text{diag}(X)$. A operação de convolução no domínio da frequência ($Z = \Lambda_x Y$) torna-se:

\[
\mathbf{Z} = 
\underbrace{
\begin{bmatrix} 
-1 & 0 & 0 \\ 
0 & 1+2\psi & 0 \\ 
0 & 0 & 1+2\psi^2 
\end{bmatrix}
}_{\text{Matriz Diagonal } (\Lambda_x)}
\begin{bmatrix} 
2 \\ 
1+\psi^2 \\ 
1-\psi 
\end{bmatrix}
\]

Executando o produto ponto a ponto:
\[
\mathbf{Z} = 
\begin{bmatrix} 
-2 \\ 
(1+2\psi)(1+\psi^2) \\ 
(1+2\psi^2)(1-\psi) 
\end{bmatrix}
=
\begin{bmatrix} 
-2 \\ 
1 + \psi^2 + 2\psi + 2\psi^3 \\ 
1 - \psi + 2\psi^2 - 2\psi^3 
\end{bmatrix}
=
\begin{bmatrix} 
-2 \\ 
-1 + 2\psi + \psi^2 \\ 
3 - \psi + 2\psi^2 
\end{bmatrix}
\]
\small{\textit{*Simplificações usando $\psi^3 = -1$.}}

\subsubsection*{Retorno ao Tempo (INTT)}

Finalmente, aplicamos a transformada inversa. O vetor resultante $z$ deve coincidir com o cálculo temporal $[-1, 2, 1]^T$.
Ao reconstruir o polinômio $z(\omega)$ a partir dos valores em $\mathbf{Z}$, obtemos os coeficientes:

\[
z = \text{INTT}(\mathbf{Z}) = \begin{bmatrix} -1 \\ 2 \\ 1 \end{bmatrix}
\]

Isso verifica que:
\[
z(\psi) = -1 + 2\psi + \psi^2
\]
O que coincide exatamente com o segundo elemento do vetor $\mathbf{Z}$ calculado acima, fechando o ciclo da prova numérica.
\subsection{Determinição unívoca do polinômio}
\label{sec:Determinição unívoca do polinômio}

\begin{teorema}[Unicidade via matriz de Vandermonde]
Sejam $x_0,\dots,x_{N-1}$ escalares dois a dois distintos em um corpo $\mathbb{K}$
(e.g., $\mathbb{R}$, $\mathbb{C}$, $\mathbb{F}_p$), e Sejam $y_0,\dots,y_{N-1}\in\mathbb{K}$.
Existe um \textbf{único} polinômio
$p(x)=a_0+a_1x+\cdots+a_{N-1}x^{N-1}\in\mathbb{K}[x]$
tal que $p(x_i)=y_i$ para todo $i=0,\dots,N-1$.
\end{teorema}

\begin{proof}
Escreva
$p(x)=\sum_{k=0}^{N-1} a_k x^k$.
Impor as condições $p(x_i)=y_i$ para $i=0,\dots,N-1$ gera o sistema linear

$$
\begin{bmatrix}
1 & x_0 & x_0^2 & \cdots & x_0^{N-1}\\
1 & x_1 & x_1^2 & \cdots & x_1^{N-1}\\
\vdots & \vdots & \vdots & \ddots & \vdots\\
1 & x_{N-1} & x_{N-1}^2 & \cdots & x_{N-1}^{N-1}
\end{bmatrix}
\begin{bmatrix}
a_0\\
a_1\\
\vdots\\
a_{N-1}
\end{bmatrix}
=
\begin{bmatrix}
y_0\\
y_1\\
\vdots\\
y_{N-1}
\end{bmatrix}.
$$

Denote essa matriz por $V$ (matriz de Vandermonde,i.e. $V_{i,j}=x_i^{j-1}$ para todo os índices i,j), o vetor de coeficientes por $a$
e o vetor de valores por $y$; então o sistema é

$$
Va = y.
$$

O determinante de Vandermonde é dado por

$$
\det(V)=\prod_{0\le i<j\le N-1}(x_j-x_i).
$$

Como os $x_i$ são dois a dois distintos, temos $x_j-x_i\ne 0$ para $i\ne j$,
logo $\det(V)\ne 0$. Portanto, $V$ é invertível e o sistema $Va=y$ tem
\textbf{solução única}, dada por

$$
a = V^{-1}y.
$$

Concluímos que existe um único vetor de coeficientes $(a_0,\dots,a_{N-1})$,
isto é, um \textbf{único} polinômio $p(x)$ de grau $\le N-1$ que interpola os $N$ pontos.
\end{proof}
\subsection{Estrutura Cíclica de Grupos Multiplicativos de Corpos Finitos}
\label{ap:fp-ciclico}

Neste apêndice, justificamos a afirmação de que o grupo multiplicativo $(\mathbb{Z}/p\mathbb{Z})^\times$ é cíclico e analisamos a existência de elementos com ordens específicas, fundamentando a existência de raízes primitivas da unidade necessárias para a definição da Transformada.

\subsubsection*{Ciclicidade de $(\mathbb{Z}/p\mathbb{Z})^\times$}

Seja $\mathbb{K}$ um corpo finito. O teorema a seguir estabelece que seu grupo multiplicativo é cíclico. No contexto do trabalho, aplicamos isso para $\mathbb{K} = \mathbb{Z}/p\mathbb{Z}$.

\begin{teorema}
    Seja $\mathbb{K}$ um corpo finito. Então o grupo multiplicativo $\mathbb{K}^\times = \mathbb{K} \setminus \{0\}$ é um grupo cíclico.
\end{teorema}

\begin{proof}
    Seja $n = |\mathbb{K}^\times|$ a ordem do grupo. Sendo $\mathbb{K}^\times$ um grupo abeliano finito, Seja $m$ a ordem máxima dentre os elementos de $\mathbb{K}^\times$. Uma propriedade fundamental de grupos abelianos finitos garante que a ordem de qualquer elemento do grupo divide a ordem máxima $m$.
    
    Portanto, para todo $x \in \mathbb{K}^\times$, temos que:
    \[
        x^m = 1.
    \]
    
    Considere agora o polinômio $f(x) = x^m - 1$ com coeficientes no corpo $\mathbb{K}$. Sabemos que um polinômio de grau $m$ sobre um corpo possui, no máximo, $m$ raízes distintas.
    
    No entanto, acabamos de ver que todos os $n$ elementos de $\mathbb{K}^\times$ satisfazem a equação $x^m - 1 = 0$. Logo, o polinômio tem $n$ raízes. Para que isso não viole o limite de raízes, devemos ter necessariamente:
    \[
        n \le m.
    \]
    
    Por outro lado, pelo Teorema de Lagrange, a ordem de qualquer elemento (incluindo o elemento de ordem máxima $m$) deve dividir a ordem do grupo ($n$). Logo, $m \le n$.
    
    Concluímos que $m = n$. Isso significa que existe um elemento em $\mathbb{K}^\times$ cuja ordem é igual à ordem do grupo. Tal elemento é, por definição, um gerador de $\mathbb{K}^\times$. Portanto, o grupo é cíclico.
\end{proof}

\subsubsection*{Existência de Elementos de Ordem $d$}

Tendo estabelecido que $G = (\mathbb{Z}/p\mathbb{Z})^\times$ é cíclico de ordem $p-1$, justificamos a recíproca mencionada no texto principal: para todo divisor da ordem do grupo, existe um elemento com aquela ordem.

\begin{proposicao}
    Seja $G$ um grupo cíclico finito de ordem $M$, gerado por $g$. Se $d$ é um divisor de $M$, então existe um elemento em $G$ com ordem exatamente $d$.
\end{proposicao}

\begin{proof}
    Como $d \mid M$, podemos escrever $M = d \cdot k$ para algum inteiro $k$. Considere o elemento $h = g^k$. Calculando as potências de $h$:
    \[
        h^d = (g^k)^d = g^{kd} = g^M = 1_G.
    \]
    Portanto, a ordem de $h$ divide $d$. Para ver que a ordem é exatamente $d$, suponha que $h^r = 1$ para $0 < r < d$. Então:
    \[
        (g^k)^r = g^{kr} = 1.
    \]
    Como a ordem de $g$ é $M$, isso implicaria que $M \mid kr$, ou seja, $dk \mid kr$, o que implica $d \mid r$. Isso contradiz $r < d$.
    
    Logo, $\operatorname{ord}(h) = d$.
\end{proof}

Aplicando ao contexto do texto principal: como $G = (\mathbb{Z}/p\mathbb{Z})^\times$ tem ordem $M = p-1$, se impusermos a condição $2N \mid (p-1)$, a proposição acima garante a existência de um elemento $\psi$ de ordem $2N$ (uma raiz primitiva $2N$-ésima da unidade).

\printbibliography

\end{document}
